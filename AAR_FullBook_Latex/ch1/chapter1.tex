\chapter{Introduction}
\label{chap1}
This chapter is from our published survey paper \cite{jinrui2024progress}.
By refueling aircraft while in flight, aerial refueling is an efficient technique to extend their endurance and range. Autonomous Aerial Refueling (AAR) is anticipated to be used to complete aerial refueling for unmanned aircraft. There are three aerial refueling methods: the probe-and-drogue method, the flying-boom method, and the boom-drogue-adapter method. The paper considers the Probe-and-Drogue Refueling (PDR) approach since its control task is the most challenging. PDR is divided into four phases, with the refueling phase being the most crucial: the rendezvous phase, joining phase, refueling phase, and reform phase. The controller design faces the most significant challenge during the docking control of the refueling phase since it calls for a high level of safety, precision, and efficiency. As a result, the modeling and control issues encountered during the refueling phase are typical and difficult. The fundamental idea of AAR is presented in the chapter initially, after which the characteristics and requirements of AAR are outlined. The development of modeling and control techniques for the AAR's refueling phase is then systematically reviewed. Besides, potential future work for high safety, precision, and efficiency requirements is examined and suggested. Finally, the objective, chapter introduction, and book structure are given. An excellent survey about modeling, sensors, control strategies, simulation, and testing can be found in \cite{AAR-2014}, which pays much attention to the practical side of AAR. In 2014, a Chinese survey \cite{quan2014survey} on PDR modeling and control was released. 

\section{Basic concept and significance of AAR}

In Aerial Refueling (AR), also known as air-to-air refueling, in-flight
refueling, or simply air refueling, one air-craft, referred to as
the tanker, refuels one or more other aircraft, referred to as the
receivers, while they are in the air. AR is frequently employed in
the military field to extend endurance, range, and resolve the conflict
between payload and takeoff distance. AR is also used in the civil
field to increase the effectiveness of long-haul flights and boost
the civil aviation's emergency response capabilities. Unmanned aerial
vehicles (UAVs) have difficulty performing AR, which is primarily
carried out by the receiver pilot in manned aircraft. Realizing autonomous
aerial refueling is important for both reducing the pilots' pressure
of performing AR and achieving the AR of UAVs while maintaining a
high enough level of safety, precision, and efficiency. 

\subsection{Classification}

Aerial refueling systems that are now in use include the flying-boom
type (Figure \ref{Fig_1.1} (a)), the probe-and-drogue type (Figure
\ref{Fig_1.1} (b)), and the boom-drogue-adapter type (Figure \ref{Fig_1.1}
(c)). The three aerial refueling systems will be detailed in the following.

A tanker with a flying boom is often modified by a civil aircraft
or a large transport aircraft. The flying boom is supervised and controlled
by a human operator from a station near the rear of the tanker aircraft.
The flying boom is often equipped with a V-tail, which can be used
to control it. The refueling process is controlled cooperatively by
the receiver and the flying boom. The receiver first flies to the
flying boom working area and then keeps relatively stationary with
the tanker. Then, the operator in the station at the rear of the tanker
controls the attitude of the flying boom, extends the flying boom
to approach the receiver slowly and insert it into the fuel tank at
the back of the receiver fuselage.

\begin{figure}
\begin{centering}
\includegraphics[width=0.9\textwidth]{Figures/Figs_Ch1/Fig_1\lyxdot 1}
\par\end{centering}
\caption{Three kinds of aerial refueling systems}

\centering{}\label{Fig_1.1}
\end{figure}

\begin{figure}
\begin{centering}
\includegraphics[width=0.4\textwidth]{Figures/Figs_Ch1/Fig_1\lyxdot 2}
\par\end{centering}
\caption{Probe-and-drogue aerial refueling apparatus}

\centering{}\label{Fig_1.2}
\end{figure}

The probe-and-drogue refueling equipment (shown in Figure \ref{Fig_1.2})
is simple and compact. A hose-drogue pod can be added to any aircraft
with enough payload capacity, including fighters, large transport
aircraft, and UAVs. The pod releases a hose with a drogue at the end,
in which the hose length should match the tanker length for safety
considerations. For manned aircraft, the probe is typically placed
to the side of the cockpit for clear pilots' visibility, but for unmanned
aerial vehicles, it is generally placed in front of the receiver nose.
The tanker frequently maintains level flight while moving forward
during the preparation phase of refueling and releases a hose in the
refueling region. The receiver then moves towards the tanker and places
its probe into the drogue. The aerial docking operation is finished
once the mechanical self-locking mechanism on the drogue locks the
drogue and the probe together. The receiver must then maintain its
position in relation to the tanker until refueling is complete.

The boom-drogue-adapter aerial refueling uses a hose installed at
the end of a flying boom. On the one hand, it is designed to keep
the flexible connection between the receiver and the tanker after
docking in order to improve the safety in the fuel transferring process.
On the other hand, it is designed to solve the shortcoming that the
rigid flying boom cannot refuel helicopters. Since the end of the
adapter is a hose, the refueling process is similar to that of the
probe-and-drogue aerial refueling. 

\begin{table}
\caption{Performance of three types of aerial refueling systems}

\begin{centering}
\begin{tabular}{|c|c|c|c|c|c|}
\hline 
Refueling type & Tanker  modification & Oil rate & Safety & Influence by disturbances & Docking difficulty\tabularnewline
\hline 
Flying-boom & ++ & ++ & + & + & +\tabularnewline
\hline 
Probe-and-drogue & + & + & +++ & +++ & +++\tabularnewline
\hline 
Boom-drogue-adapter & ++ & + & ++ & ++ & ++\tabularnewline
\hline 
\end{tabular}
\par\end{centering}
\centering{}\label{Tab_1.1}
\end{table}

The three different types of aerial refueling systems' primary performance
is displayed in Table \ref{Tab_1.1}. It can be seen from Table \ref{Tab_1.1}
that the boom-drogue-adapter type has most of the features of the
flying-boom refueling and is slightly compromised towards the probe-and-drogue
type. Because of its flexible connection and lightweight equipment,
the PDR has a wide range of applications despite its challenging docking.
By mounting multiple refueling pods on a receiver, PDR enables multi-point
refueling, namely simultaneous refueling of multiple aircraft. PDR
can also support aerial refueling for helicopters. The United States
Navy and the air forces of many countries favor the probe-and-drogue
refueling method due to its adaptability to various aircraft and refueling
rates. In contrast, airborne early warning aircraft and some fighters
use the flying-boom refueling method. When refueling numerous aircraft,
PDR is favored if small aircraft predominate in the refueling process
because small aircraft have lower oil transfer rate requirements,
and PDR can accomplish multi-point refueling. If large aircraft outnumber
small aircraft, flying-boom refueling may be chosen instead. As a
result, from the implementation perspective, one refueling method
cannot replace another. The PDR is primarily considered in this book.

\subsection{Aerial refueling phase}

\begin{figure}
\begin{centering}
\includegraphics[width=0.5\textwidth]{Figures/Figs_Ch1/Fig_1\lyxdot 3}
\par\end{centering}
\caption{Aerial refueling phase}

\centering{}\label{Fig_1.3}
\end{figure}

Aerial refueling consists of four phases: rendezvous phase, joining
phase, refueling phase and reform phase. The details are as follows:

(i) Rendezvous phase (to $A$ shown in Figure.\ref{Fig_1.3}): The
process of the rendezvous between the tanker and the receiver flying
from different directions to the designated area. For a single receiver,
it waits at point $A$ (as shown in Figure. \ref{Fig_1.3}) for the
next instruction. 

(ii) Joining phase ($A$-\textgreater$B$ -\textgreater$B^{\prime}${}
shown in Figure.\ref{Fig_1.3}): The process of the receiver entering
the observation area, completing the formation with the tanker and
other receivers. As shown in Figure.\ref{Fig_1.3}, for multiple receivers,
they will be in a staggered queue in the observation area and move
from $B$ to $B^{\prime}$ one by one. 

(iii) Refueling phase ($B^{\prime}$ -\textgreater$C$ -\textgreater{}
$C^{\prime}$ shown in Figure.\ref{Fig_1.3}): The process of the
receiver from the observation area to the astern area. After a successful
capture, the receiver and the tanker keep relatively stationary to
transfer the fuel. There are two main subphases, namely docking subphase
and fuel transferring subphase, where the former is especially important
and challenging. 

(iv) Reform phase ($C^{\prime}$ -\textgreater$D${} shown in Figure.\ref{Fig_1.3}):
After completing the fuel transferring, the probe in the receiver
and the drogue are separated, and then the receiver leaves the astern
area and enters the reform area. 

\subsection{AAR}

With the development of aerial refueling technology and the increasing
requirements of aerial refueling from modern flight missions, aerial
refueling urgently needs to be autonomous to expand its scope of application
and meanwhile be implemented with high precision, high safety, and
high efficiency. The importance and difficulties of AAR were comprehensively
discussed in \cite{Nalepka-2005-1}. AAR refers to automating
manned or unmanned aerial refueling. Under a specific autonomous authority,
the receiver automatically processes sensor data and creates relevant
trajectory commands for its guidance and control system to implement
AAR. 

On April 22, 2015, Northrop Grumman Corporation and the U.S. Navy
successfully demonstrated full probe-and-drogue AAR with an X-47B
UAV for the first time in history that a UAV has been refueled in-flight
(as shown in Figure.\ref{Fig_1.4} (a)). Before it, two small UAVs
flying in leader-follower formation performed airborne docking by
the University of Sydney. A complete navigation, guidance and control
solution was given in \cite{wilson2015guidance}. On May 09, 2017,
Airbus Defense and Space successfully demonstrated AAR contacts between
a fighter aircraft and a tanker\textquoteright s flying boom for the
first time in the world (as shown in Figure.\ref{Fig_1.4} (b)). Airbus
has disclosed that the testing for a new function that would enable
the A330-200 Multi Role Tanker Transport (MRTT) to achieve AAR for
UAVs will begin in 2023.

\begin{figure}
\begin{centering}
\includegraphics[width=0.8\textwidth]{Figures/Figs_Ch1/Fig_1\lyxdot 4}
\par\end{centering}
\caption{AAR for the first time}

\centering{}\label{Fig_1.4}
\end{figure}


\subsection{The significance of AAR}

Aerial refueling is widely used in the military field, such as the
endurance of fighters and the take-off with a heavy load of carrier-based
aircraft. Besides, the technology is being actively promoted to the
civil field to reduce fuel load and improve the economy of civil aviation
\cite{bennington2005aerial}. 

For military fields, the meaning of aerial refueling is significant. 

(i) Increasing the range. The operational radius is one of the most
important indicators to measure the combat capability of the fighter
and even the air force. After a single aerial refueling, the bomber's
operational radius can be increased by 25\%-30\%, the fighter's operational
radius can be increased by 30\%-40\%, and the range of the transport
aircraft can be doubled. 

(ii) Prolonging the endurance. Patrol aircraft, early warning aircraft,
reconnaissance aircraft, and other special duty aircraft often need
to stay in the air for a long time. The use of aerial refueling can
prolong the endurance and avoid the delay and inconvenience caused
by the landing for the refueling. 

(iii) Solving the contradiction between payload and take-off distance.
The short-distance runway, such as that for an aircraft carrier and
a plateau airport, may not meet the requirement of full-payload takeoff.
Therefore, the aircraft can take off with a part of the fuel and then
be refueled by a tanker in the air. 

For civilian areas, the benefits of aerial refueling are in the following. 

(i) Making long-distance flights more efficient. Avoid landing and
taking off at midway airports for fuel supply and make long-distance
flights more efficient. 

(ii) Improving the emergency response capability of civil aviation.
Refuel civil aviation in case of emergency and increase endurance. 

The main purpose of AAR is to extend the aerial refueling technology
to the UAV field, and it can also be used to assist docking operation
to reduce the burden of pilots. The docking phase of aerial refueling
is a high-precision and high-risk maneuvering flight phase. The improper
operation will not only lead to the docking failure, but also even
lead to serious flight accidents caused by the damage to the trailing
hose and drogue. Therefore, it is necessary to standardize the automatic
operation procedure to ensure the safe and efficient operation of
aerial refueling. For UAVs, payload and endurance are important problems
for UAV development. In the \textquotedblleft Unmanned Aircraft Systems
Roadmap 2005-2030\textquotedbl{} published by the US Air Force in
2005 \cite{cambone2005unmanned}, AAR is regarded as an important
way to solve these problems, and the technology was expected to be
realized for UAVs in 2015-2020. 

\subsection{The significance of docking phase of probe-and-drogue AAR }

AAR mainly emphasizes the autonomy of the receiver, namely getting
rid of or minimizing human participation of the whole process from
the rendezvous phase to the reform phase. From this point of view,
it is more general to study the autonomous docking. First of all,
the docking control problem of the probe-and-drogue refueling is more
representative. The flying boom is rigid, which is hardly disturbed
by the airflow. By contrast, the hose with a drogue is flexible, which
is easy to be disturbed by the airflow. Thus, probe-and-drogue AAR
faces more difficult docking problems. Secondly, although the flying-boom
refueling requires less on receivers and is easier to achieve aerial
refueling for manned aerial vehicles, the advantages of the flying-boom
refueling are not obvious for the development of the UAV-to-UAV aerial
refueling. Probe-and-drogue AAR only needs to control the receiver
to achieve the UAV-to-UAV aerial refueling, while the flying-boom
refueling needs the tanker to cooperate, i.e., it also needs to design
the autonomous control for the flying boom to achieve the UAV-to-UAV
aerial refueling. 

Among the four phases of aerial refueling, the refueling phase, especially
the docking subphase, is the most difficult phase, which is representative
and challenging. It requires the highest accuracy, safety, and efficiency,
and the key to the success of aerial refueling is to solve the control
problem of this phase. In addition, the control problem of the refueling
phase covers position keeping and trajectory tracking, which is also
a problem that needs to be solved in all phases. 

\section{Characteristics and problems of the docking control of probe-and-drogue
AAR \label{sec:Characteristics-and-problems}}

Before introducing modeling and control, it is necessary to understand
the characteristics and requirements of the docking control of probe-and-drogue
AAR, which will help to understand the current modeling and control
methods better. 

\subsection{Characteristic analysis of probe-and-drogue AAR }

\subsubsection{Uncertainty }

(i) The uncertainty of the motion of the receiver caused by the tanker
vortex and wind turbulence. Aerial refueling is a close formation
flying mission. The tanker in front generates a vortex, which affects
the flight performance of the receiver behind. The vortex of the tanker
can be divided into three types: the wake vortex (also known as the
wingtip vortex), the engine jet and the turbulent flow in the boundary
layer, wherein the wake vortex has the greatest influence on the receiver.
The wake vortex is formed by the airflow from the bottom of the tanker
wingtip. Before forming, the vortex is similar to downwash flow, and
after forming, it becomes two downward and backward spiral flows pulled
by the wingtip. Besides the tanker vortex, the receiver will also
be affected by atmospheric turbulence and wind gust. 

(ii) The uncertainty of the drogue motion. Wind gust and tanker wake
vortex may cause the drogue to drift, which makes docking position
very uncertain. Another important influence is brought by the receiver.
In high-speed flight, the drogue is affected by the airflow change
near the receiver nose, which is called the bow wave effect \cite{Dogan-2013-7},
\cite{bhandari2013bow} or forebody (aerodynamic) effect \cite{Dibley-2007-2},
\cite{ro2011dynamics}. When approaching the receiver nose, the drogue
will be pushed away from the original position. This disturbance is
evident when the probe is on the side of the receiver\textquoteright s
central symmetrical plane, which is a widely-used configuration as
shown in Figure \ref{Fig_1.5} (a). At present, some reported UAVs
with aerial refueling capability adopt the \textquotedblleft on the
nose\textquotedbl{} configuration as shown in Figure \ref{Fig_1.5}
(b). When the probe is \textquotedblleft on the side\textquotedblright ,
the drogue will pass the nose, so the bow wave effect is very serious.
In contrast, the effect of the bow wave effect of the receiver with
the probe \textquotedblleft on the nose\textquotedbl{} configuration
is not as obvious as that of the probe on the side. This is because
the drogue will not pass the nose for docking. 

(iii) The uncertainty in the dynamic model of the receiver. As for
the accurate docking requirements, the flight control model of the
receiver with parameters generated by the wind tunnel has some uncertainties
which cannot be ignored. In the process of refueling, the mass of
the receiver will increase with the increase of the amount of fuel.
Meanwhile, the mass of the tanker will decrease. What is more, for
both the tanker and the receiver, their positions of the center of
mass will change during refueling. It should be noticed that a larger
mass proportion will increase for the receiver because it is often
lighter than the tanker. As a result, uncertainty in the dynamic model
of the receiver will be correspondingly greater. In addition, because
of the highly accurate docking requirement, some uncertainties on
the control actuator cannot be ignored as well, such as throttle thrust
accuracy. All of these will lead to uncertainty in the dynamic model
of the receiver. 

\begin{figure}
\begin{centering}
\includegraphics[width=0.5\textwidth]{Figures/Figs_Ch1/Fig_1\lyxdot 5}
\par\end{centering}
\caption{Probe positions}

\centering{}\label{Fig_1.5}
\end{figure}

(iv) The uncertainty of initial states. Before docking, the receiver
has to complete the formation with the tanker. They both keep a constant
speed and fixed altitude. The receiver at a position close to the
tanker keeps relatively stationary with the tanker. Then, the receiver
will take this position as the initial position to start the docking
task. However, due to various uncertain factors, the receiver itself
cannot accurately stay at this position but floats in a small range.

\subsubsection{Slower dynamic}

During the docking process, the flow filed of the receiver may push
the drogue away as the receiver approaches the tanker. Because the
mass and the volume of the drogue are smaller than that of the receiver,
it is easy to be disturbed by disturbances and then swings fast. The
swing frequency is so high that the dynamic response of the receiver
is hard to track. Not only that, but the receiver also needs to change
attitude indirectly to change the direction of velocity. These lead
the receiver has a slower dynamic. In the experimental paper \cite{Dibley-2007-2}
provided by the National Aeronautics and Space Administration (NASA),
the author explained that in the actual aerial refueling experiments,
the velocity changes of the receiver\textquoteright s probe in the
vertical direction and lateral direction obviously lagged behind the
drogue. This problem makes it difficult for receivers to capture a
fast-moving drogue by only using conventional feedback methods. In
order to solve this problem, some studies focused on increasing the
damping of drogue motion to slow down the drogue dynamic, such as
adding an active stability control device \cite{Ro-2010-5}.

\subsubsection{Nonminimum Phase \cite{hoagg2007nonminimum}}

For a fixed-wing aircraft with the conventional configuration, the
aerodynamic focus is designed behind the center of gravity in order
to make the aircraft have good static stability. If the aircraft wants
to move in one direction, it must be put a force in the opposite direction
to adjust its attitude, so its flight trajectory will move in the
opposite direction for a short period before starting to move in the
desired direction. This causes the nonminimum phase system to adjust
for a long time, making it unable to track changing trajectories rapidly.
Thus, trajectory tracking becomes more difficult due to the nonminimum
phase property of the receiver. Mathematically, the nonminimum phase
property for linear systems means that the transfer function has unstable
zeroes, i.e., zeroes with positive real parts, which means that the
inverse transfer function are unstable. Moreover, a given non-minimum
phase system will have a greater phase contribution than the minimum
phase system with the equivalent magnitude response. For an aircraft
with the configuration shown in \cite{benvenuti1994approximate},
since the positive real part of the zero of the longitudinal channel
transfer function is very large, the unstable zero can be ignored
in most cases. However, it cannot be ignored in the docking control
as the requirement of the control accuracy is very high. 

\subsection{Requirements in the docking process }

In the refueling phase, the docking procedure is difficult and demands
a high level of safety, precision, and efficiency. Table \ref{Tab_1.3}
lists the three requirements along with the justifications for the
requirements' necessity. 
\begin{table}
\caption{Requirements in the docking process}

\begin{centering}
\begin{tabular}{|c|>{\centering}p{2.5cm}|>{\centering}p{5cm}|>{\centering}p{5cm}|}
\hline 
\textbf{Requirements} & High Safety & High Precision 

(Position error $\leq$0.2/0.25/0.3m; 

Velocity error: 1\textendash 1.5m/s)  & High Efficiency\tabularnewline
\hline 
\textbf{Justifications} & Receiver flies close to the tanker; 

Probe collides with the drogue  & Drogue radius is 0.305m; 

Too low relative speed cannot hit the valve open; 

Too high relative speed will damage the drogue.  & Receiver becomes vulnerable to attacks and anti-reconnaissance capability
decreases during AR;

To save time for the primary mission \tabularnewline
\hline 
\end{tabular}
\par\end{centering}
\centering{}\label{Tab_1.3}
\end{table}


\subsubsection{High Safety}

In an actual aerial refueling process, the most important thing is
to reduce the risk and ensure flight safety, after which the success
of docking can be considered. With the traditional flight knowledge,
it is dangerous for aircraft to collide with an object, e.g. a bird,
in the air. When refueling in the air, the receiver pilot needs to
control the aircraft to make the receiver\textquoteright s probe collide
with the drogue within the prescribed range of relative velocity for
docking. In this process, risk detection and risk control are needed
to reduce risk consequences and the frequency of their occurrence.
To be specific, the receiver should first avoid colliding with the
tanker in the air. Furthermore, the apparatus (hose, drogue, probe
or other refueling equipment) damage should be avoided, which requires
that the capture speed control should be reasonable in the docking
process, and the offset distance of the drogue should be strictly
controlled in the docking process. Once the relative velocity and
the docking error between the drogue and the probe exceed reasonable
thresholds (specified safety range requirements), the receiver is
required to return to a safe place to adjust and evaluate whether
to continue docking or not to ensure the safety of the docking.

\subsubsection{High Accuracy}

The docking control accuracy of probe-and-drogue aerial refueling
methods needs to be within the centimeter level \cite{doebbler2007boom},
which is a quite high accuracy requirement for actual aircraft control
system due to aircraft maneuverability limitation and atmospheric
disturbances. This requires that the navigation and control accuracy
reaches the centimeter level or higher. Moreover, the drogue-and-probe
refueling also needs to limit the capture speed within a very small
range, such as 1\textendash 1.5m/s \cite{AAR-2014}. This kind of
speed can cause the probe to hit the fuel transferring valve in the
drogue open. Too small speed cannot hit the valve open, and too high
speed will damage the drogue.

\subsubsection{High Efficiency}

In the aerial refueling process, especially in the docking and fuel
transferring phase, the receiver\textquoteright s concealment and
anti-reconnaissance capability will decrease greatly. Once be discovered,
it will become the enemy's key target, so the aerial refueling must
be completed within a short time. This is particularly necessary in
the enemy-occupied airspace. In addition, for both military and civilian
fields, aerial refueling as an auxiliary mission aims to achieve the
main task better, so it also needs a more efficient implementation
on the basis of safety. 

\section{Progress in Modeling of PDR}

There are two purposes of mathematical modeling for the PDR system:
one is for high-fidelity analysis of the aerial refueling system through
virtual flight simulations with computers, and the other is for the
controller design. Generally, the former tends to establish a very
elaborate model to include all the kinematics and dynamics of the
PDR system, while the latter is based on a simplified model, which
is often obtained by simplifying the former model to some degree.
The modeling of PDR systems consists of mathematical descriptions
of the kinematics and dynamics of aircraft, refueling equipment (hose,
drogue, probe), and wind disturbances.

\subsection{Aircraft modeling}

Tanker modeling and receiver modeling are both a part of aircraft
modeling. In the PDR, the receiver moves to carry out a refueling
procedure while the tanker simply flies level and forward. As a result,
studies are mainly concentrated on receiver modeling. 

In most studies, the tanker model is frequently regarded as a mass
point model \cite{fezans2018towards}. In addition, a rigid model
with six degrees of freedom (DOF) is created in order to build a controller
for the tanker. A strict-feedback version of a non-affine nonlinear
tanker model was taken into consideration in Ref. \cite{su2016robust}. 

Linear models with small-angle perturbations are typically used for
receiver modeling\cite{brian2003aircraft}. High model precision is
necessary for aerial refueling, and it is also necessary to characterize
the model's dynamic properties under the wake vortex. The receiver
model in Ref.\cite{bloy2001modeling} was modified for this reason.
The docking model of a tailless UAV was put forth in Ref.\cite{barfield2005equivalent}.
The receiver modeling with varying mass was examined in Refs.\cite{guo2011modeling}
\cite{an2018relative}. Ref.\cite{wu2021dynamic} took into account
the modeling of the time-varying uncertain inertia. For the ease of
the latter nonlinear controller design, nonlinear receiver models
were also taken into consideration in the previous couple of years
and frequently converted to an affine nonlinear form\textbf{ }\cite{su2018exact}. 

\subsection{Refueling apparatus modeling}

The primary components of PDR equipment are a hose, a drogue, and
a Hose-Drum Unit (HDU). The flexible hose causes the drogue not to
be fixed in the position under the tanker coordinate system. Studying
the basic dynamics of the drogue is helpful to do the simulation and
design the docking control law.

The cornerstone of PDR equipment modeling is hose modeling. Since
the hose is flexible, the lumped parameter technique and the finite-segment
strategy are frequently used to create a link-connected model\cite{bloy2002modelling}
\cite{zibo2012research}. Figure \ref{Fig_1.6} illustrates the breakdown
of a hose into a number of lumped-mass, rigid cylindrical links joined
by frictionless ball-and-socket joints. Each link is subject to aerodynamic
and gravitational loads brought by the tanker wake, disturbances in
the atmosphere, etc. The link masses and all external forces are lumped
at the connecting joints. The coordinate system in Figure \ref{Fig_1.6}
is anchored to the tanker. In Ref.\cite{hose-link-model}, simulations
demonstrated a reasonable agreement between the model's features and
the published flight test data. A basic pendulum-based hose model
was also researched in Refs. \cite{pud-drogue,dai2018iterative}.

\begin{figure}
\begin{centering}
\includegraphics[width=0.5\textwidth]{Figures/Figs_Ch1/Fig_1\lyxdot 6}
\par\end{centering}
\caption{Link-connected hose model ($l_{*}$ indicates the link length, $p_{*}$
indicates the position of each joint).}

\centering{}\label{Fig_1.6}
\end{figure}

Ref.\cite{vassberg2002dynamic} performed a dynamic study of the KC-10
tanker's hose but neglected to account for two important elements:
restoring force between two rigid links and variable length. Ref.\cite{yin2022hose}
investigated the attitude model of the hose-drogue system while taking
into account restoring force by combining CFD and multi-rigid body
dynamics. After that, the hose-drogue system's movement rule was examined.
This force was added to the hose model in Ref.\cite{HoseLink-2007}.
To prevent harm to the equipment, the model can be used to evaluate
the Hose Whip Phenomenon (HWP) \cite{ro2011dynamics}. By adjusting
the hose length to keep the hose tension, the HDU hose reel control
device can prevent HWP. In Ref.\cite{vassberg2003numerical}, a modeling
approach for the variable-length hose was proposed. In Ref. \cite{wang2014dynamic},
a dynamic variable-length hose-drogue model with the restoring force
resulting from bending was examined. An integrated drogue model was
suggested in Ref. \cite{dai2019hose} to explain the drogue behavior
under wind disturbances while taking the impact of the HDU controller
into consideration. Additionally, system identification was used to
create a reduced lower-order drogue dynamic model that aided in constructing
the docking controller. 

The core of the lumped-mass finite-segment technique is rigid link
kinematics, which did not take into account the hose's elasticity
and damping. The elastodynamic hose model was created using the finite
element method (FEM) \cite{zhu2006elastodynamic}, which is based
on the mechanics of materials. A three-node, nonlinear curved beam
element was used to model the hose in the Lagrangian framework \cite{zhu2007modeling},
and the dynamic analysis of the established model regarding cable
tension, tow point disturbance, and vortex wake was also investigated. 

The hose was discretized in the study mentioned above to model the
hose-drogue assembly. As shown in Figure \ref{Fig_1.7}, some research,
in contrast, established continuum models based on the analysis of
infinitesimals . The hose is considered as an infinite dimensional
distributed parameter system, and the dynamic equation is built using
partial differential equations. Ref. \cite{paniagua2021aeroelastic}
suggested a precise hose-drogue model that took into account the hose
bending effect, the downwash angle, and the phase lag between the
hose motion and its unsteady aerodynamic forces in order to study
the flutter-type aeroelastic instability of the hose-drogue system.
Ref. \cite{liu2020vibration} used partial differential equations
to model the flexible hose as an infinite dimensional distributed
parameter system to prevent spillover effects caused by truncated
hose models. 

\begin{figure}
\begin{centering}
\includegraphics[width=0.5\textwidth]{Figures/Figs_Ch1/Fig_1\lyxdot 7}
\par\end{centering}
\caption{Continuum model established on the analysis of \cite{paniagua2021aeroelastic}
( $T$ is the hose tension, $p$ and $q$ are the hose drag and lift
per unit length, $\rho_{H}$ is the hose mass per unit length).}

\centering{}\label{Fig_1.7}
\end{figure}

A full hose-drogue assembly model is produced after the hose model
is established and the aerodynamic model of the drogue is combined.
Ref. \cite{wei2016drogue} extracted a second-order transfer function
model to characterize drogue dynamics under the bow wave effect by
parameter identification of the higher-order link-connected system
in order to create a lower-order dynamic model suitable to the latter
docking controller design.

\subsection{Wind disturbance modeling}

During the refueling process, the trailing hose-drogue and the receiver
are subject to various disturbances caused by the airflow, thus increasing
the difficulty of control. The main wind disturbances can be classified
into three categories: (i)the bow wave effect from the receiver, (ii)
the turbulence and wind gust in the atmosphere, and (iii) the wake
vortex caused by the tanker. The receiver is affected by the last
two disturbances, while the hose-drogue system is affected by three
disturbances.

\subsubsection{Bow wave effect. }

One of the greatest issues for PDR is the disturbance of the drogue
brought on by the bow wave of the receiver, which has received a lot
of focus in the past ten years\cite{dai2018terminal}. The bow wave
effect or forebody (aerodynamic) effect as shown in Figure \ref{Fig_1.8-7}
occurs during aerial refueling because the aircraft frequently travel
at high speed, creating a powerful airflow at the receiver nose\textbf{
}\cite{ro2011dynamics,Dibley-2007-2}. As the receiver reaches the
tanker, the airflow will disturb the drogue's motion. This kind of
disturbance is more obvious when the probe is side-mounted on the
aircraft forebody, which is widely used in manned aircraft for pilots
to better observe the relative position of the probe and the drogue,
but it still exists for the probe \textquotedblleft on the nose\textquotedblright{}
although it has less influence. In reality, a docking failure is frequently
caused by this disturbance. According to NASA experimental data from
2004\cite{vachon2004calculated}, \cite{Hansen-2004-4}, as the receiver
gets closer to the drogue, it typically produces a location offset
of 30.5\textendash 36.6 centimeters from the desired position. Because
of the light weight of the drogue, it moved very fast in the docking
process, caused overshooting, and then swang back. If the receiver
was forced to connect with the drogue, then the receiver may have
a significant overshoot and cause the probe to break \cite{ro2011dynamics}.
Therefore, in order to finish docking, it is essential to anticipate
the drogue offset. Based on their previous experience, pilots frequently
predict the drogue offset. Similarly, it is expected that the receiver
autopilot will anticipate the drogue offset to finish AAR. In other
words, it is important to examine the relationship between the force
acting on the drogue and its relative location and speed with respect
to the receiver nose. 
\begin{figure}
\begin{centering}
\includegraphics[width=0.6\textwidth]{Figures/Figs_Ch1/Fig_1\lyxdot 8\lyxdot 1}
\par\end{centering}
\caption{Bow wave effect\textbf{ }\cite{bhandari2013bow}}

\centering{}\label{Fig_1.8-7}
\end{figure}

Since the bow wave is very complicated, it is challenging to develop
an extremely precise dynamic model. A lot of computer simulation data
and flight experiment data are needed, which will use up a lot of
material and human resources. It is also essential to think about
what kind of model to construct. The current modeling techniques are
split into offline and online methods depending on whether real-time
data processing is used. The acquired models are divided into dynamic
and static models depending on whether the model is dynamic. 

\textbf{i) Offline dynamic model.} If the probe of the receiver is
on the side of the nose, it is necessary to consider the influence
of the nose on the drogue in the modeling and establish the dynamic
model. In Ref. \cite{Dogan-2013-7}, Cart3D analyses were used to
study the bow wave of a C-141 receiver behind a KC-135 tanker, and
an analytical approach based on stream functions was suggested. The
resulting nonuniform flow field was then roughly modeled using components
of uniform velocity components. Based on the stream function technique,
Ref.\cite{dai2016modeling} developed an analytical bow wave model
for PDR that took the receiver nose and cockpit into account. The
analytical model is applicable to the latter controller design and
real-time simulations. Ref. \cite{wei2016drogue} used training data
from CFD models to create a nonlinear bow wave model for PDR using
nonlinear regression. 

\textbf{ii) Offline static model.} Current modeling techniques typically
focus on the entire dynamic AAR process. However, for docking control,
the trajectory of the drogue to the steady-state value is less important
than the ultimate steady-state drift value of the drogue produced
by the bow wave. In this way, a simple static model can be built,
and the parameter identification becomes much simpler. As a consequence,
a straightforward static model can be created, such as lookup tables
based on CFD analysis or qualitative static findings from trials.
According to the aerial refueling experiments conducted by NASA using
F/A-18 fighter aircraft, two conclusions can be drawn. First, the
steady position of drogue has a functional relationship with the tanker\textquoteright s
altitude and velocity; secondly, the drogue drift offset when it is
disturbed by the receiver has some certain rules, where the drogue
offset is that of the drogue from its steady-state position. That
is to say, although the airflow environment is complex, the steady
law of drogue motion can still be described by modeling. Further,
NASA conducted research on the correlation between the drogue offset
under disturbances and the receiver location. The receiver shifted
in the perpendicular plane during NASA's trial\cite{Hansen-2004-4}.
The drogue's greatest steady-state deviation while being affected
by the bow wave was approximately 0.368m. Reference \cite{Ro-2010-5}
gave similar conclusions by modeling and simulation. The basic drogue
dynamic was given in \cite{zibo2012research}.

\textbf{iii) Online model.} In practice, the pilot's operation follows
``aligning $\rightarrow$ docking $\rightarrow$ gaining experience
$\rightarrow$ returning to the initial docking position $\rightarrow$
realigning...\textquotedblright . The iterative process is carried
out in the aerial docking. In actual flight, online identification
corresponding to \textquotedblleft gain experience\textquotedbl{}
can also be considered for autonomous docking in the air. The online
identification mentioned here is performed after one docking trial,
which can use all the data from the last docking trial. The resulting
model can be used for the next trial. Compared with real-time identification
at each sensor sampling period, the identification here can use more
sampling data and has sufficient calculation time. There aren't many
findings for the online bow wave model. In order to provide online
static models, some learning-based strategies have been attempted.
To guarantee the precision of the docking control, Ref. \cite{liu2018deep}
proposed the deep learning technique to model the bow wave and predict
the location of the drogue in real-time. Additionally, the drogue
offset was predicted online in Refs.\cite{dai2018iterative,dai2018terminal}. 

\subsubsection{Atmospheric turbulence and wind gust}

Nearly all aircraft must contend with atmospheric turbulence and wind
gusts, and related models are very rich. Corresponding models are
very rich. At present, the Dryden turbulence model proposed by NASA
is widely used in the study of the aerial refueling \cite{burns2005automated}.
The realization is that the band-limited white noise passes through
specific forming filters to obtain the longitudinal, lateral and vertical
velocities of the atmospheric turbulence flow field and their angular
velocities. There are two widely accepted models: the model with MIL-HDBK-1797
standard \cite{specification1997flying} and the model with MIL-F-8785C
standard \cite{specification1980flying}. 

\subsubsection{Wake vortex caused by the tanker}

The wake vortex caused by the tanker received the most attention by
researchers among the three kinds of disturbances. When the relative
distance is short, large aircraft create a powerful wake vortex at
the wingtips that progressively diffuses backward and leaves a strong
aerodynamic interference to the aircraft behind, as shown in Figure
\ref{Fig_1.8} . Due to the closer relative distance between a tanker
and a receiver during aerial refueling compared with typical formation
flight, the wake vortex's impact is more severe\cite{venkataramanan2003vortex}.
In 2004, the computer result was contrasted with the wind tunnel experimental
result in order to assess the viability of wake vortex modeling\cite{blake2004uav}.
Although there was a good agreement between the two findings, drag
and peak lift were not well congruent. As a result, from 2004 to 2008,
the modeling of wake vortices was thoroughly studied\cite{dogan2008wakeA}
\cite{dogan2008wakeB} and represented by a more cohesive mathematical
model that divided the wake vortex's action into equivalent wind components
and wind gradient components along three-axis directions. The wake
vortex velocity was determined by the combination of the two components.
Rankine Vortex, Lamb-Oseen Vortex, and Hallock-Burnham Vortex are
the three most frequently used vortex models at present\cite{gerz2002commercial},
with Hallock-Burnham Vortex being used in aerial refueling the most.

A horseshoe vortex model was established in Ref.\cite{Vortex-1} using
the lifting line theory. Ref.\cite{zhang2020tanker} investigated
tanker wake effects through ANSYS. It was observed that wingtip vortices
of the tanker wing and horizontal tail dominate the tanker wake effects
on the receiver. Ref.\cite{cavallo2019low} developed a lower-order
analytic response surface model based on three-dimensional tricubic
interpolation to incorporate an aerodynamic interference module into
the simulation of an aircraft. In this model, the complex wake effects
were described by an increase in the aerodynamic coefficient. Additionally,
actual flight data was used to verify the model. Based on the CFD
technique\cite{yue2016numerical}, a numerical study of the flow field
of an embarked aircraft being refueled by a buddy aircraft was produced.
In order to simulate the aerodynamic interplay from the tanker to
the receiver, the hose, and the drogue, Reynolds-Averaged Navier Stokes
CFD calculations were carried out in Ref.\cite{fezans2018towards}.
Ref.\cite{katz2017aerodynamic} established an inviscid-flow-based
model to estimate the aerodynamic interaction between a tanker and
a much smaller receiver.
\begin{figure}
\begin{centering}
\includegraphics[width=0.5\textwidth]{Figures/Figs_Ch1/Fig_1\lyxdot 8}
\par\end{centering}
\caption{Wake-vortex effect \cite{Vortex-1}}

\centering{}\label{Fig_1.8}
\end{figure}


\subsection{Modeling and validation methods}

There are generally four kinds of modeling methods: first principles
modeling methods (mechanism modelling method), computational fluid
dynamics (CFD) modeling methods, wind tunnel modeling methods, and
real experimental modeling methods. The first principles modeling
method can give a unified model well, but model parameters are uncertain
or too many, which need to be determined by identification or measurement.
The CFD method is commonly used in aerodynamics modeling and simulation,
although the amount of calculation is huge. It is economical to conduct
numerical experiments on the fluid dynamic problems of the aerial
refueling docking phase to get data. Through the data, the parameters
in the model established by the first principles can be corrected
or be fitted. For example, the CFD method is used in the analysis
of bow wave models in \cite{Dogan-2013-7}. Compared with the former
two methods, the wind tunnel modeling method has a higher authenticity
but a higher cost, which can be further used to improve the existing
models. Due to the limitation of experimental conditions, the wind
tunnel experiments in the present study only aim at some special parts
of aerial refuelings, such as the aerodynamic characteristics of the
drogue \cite{ro2006aerodynamic}. The modeling methods above can be
used for the modeling of the receiver, the modeling of the trailing
hose and drogue, and the modeling of the bow wave effect on the drogue.
As for the docking process of aerial refueling, existing real experimental
modelings from the open research and academic publications were generally
used to verify or carry out some simple model modeling, such as the
wake vortex model and the maximum drogue offset under the bow wave.

\section{Progress in Control of PDR}

The primary control challenge during the refueling phase is docking
the probe into the drogue and maintaining the relative position. The
goal of docking control is to effectively guide the probe to successfully
catch the drogue by having the receiver approach the drogue along
a specific trajectory. Figure \ref{Fig_1.8-2} illustrates how docking
controllers are typically composed of three terms: the command generator
term, the tracking control term, and the stabilizing control term.
The purpose of the command generator is to ensure the smoothness of
the tracking reference and further ensure the control performance.
The purpose of the tracking control term is to design a feedforward
to turn the tracking problem into a stabilizing control problem, thus
solving the nonminimum phase problem. The purpose of stabilizing control
is to suppress some uncertainties and disturbances. Additionally,
the station-keeping control must be finished both before and after
docking. It could be viewed as a special case of tracking control.
Along with receiver control, anti-HWP control (HDU control) and drogue
stabilization and control (control from the control surfaces of the
drogue) are also attractive. Without taking into account the motion
of the drogue, the control issue in the flying-boom aerial refueling
can be basically viewed as a special case of the PDR control issue.
\begin{figure}
\begin{centering}
\includegraphics[width=0.9\textwidth]{Figures/Figs_Ch1/Fig_1\lyxdot 8\lyxdot 2}
\par\end{centering}
\caption{Overall structure of the AAR system}

\centering{}\label{Fig_1.8-2}
\end{figure}


\subsection{Command generator}

For the receiver to successfully complete the formation and refueling
tasks in a secure manner, the command generator primarily plans a
smooth and practical flight trajectory. There are two reasons for
this.

(i) Avoid overshooting. Fixed-wing aircraft often possess nonminimum
phase characteristics. In the time domain, nonminimum phase characteristics
may cause overshooting, and the amplitude of overshooting is related
to the amplitude of the input signal. Therefore, if a far-reaching
docking target is directly fed, then a large resulting signal change
will cause a large overshooting. Smooth trajectories can weaken the
effect of overshooting. In this way, the collision between the receiver
and the drogue is avoided while docking in the air. Furthermore, the
collision between the receiver and the tanker caused by overshooting
can be avoided.

(ii) Avoid induced oscillation. If a large relative position error
is directly feedback into the position tracking controller, the controller
may produce a large amount of control command, resulting in actuator
saturation (the control input value exceeds the actuating capacity
or limit of the actuator), which may induce oscillation. Even without
induced oscillation, the trajectories may be too aggressive. Docking
control in aerial refueling is a high-precision task, but many control
methods depend on linear models near the equilibrium point. Large
control signals may cause the system to move away from the equilibrium
point into nonlinear regions, resulting in that the control accuracy
reduces significantly. Therefore, reasonable trajectories must be
generated to ensure that the controller command is within acceptable
limits.

In this manner, aerial docking is possible without the receiver and
drogue/tanker colliding. Various strategies are used to accomplish
these objectives, and they can be categorized into five groups.

\textbf{Method 1: Low pass filter method.} A simple method is to make
difference between the reference trajectory (the command signal generated
by the desired position) and the measured receiver position, and then
feed the error signal into a low-pass filter (LPF) \cite{valasek2002vision}
\cite{kimmett2002vision}, \cite{kimmett2002autonomous}, \cite{wang2010verifiable},
\cite{wang2008novel}, \cite{fravolini2003development}. For example,
the low-pass filter used in references \cite{valasek2002vision} and
\cite{kimmett2002autonomous} consists of a Proportional-Integral
(PI) term and is incorporated directly into the design of the tracking
controller. On this basis, in order to track the moving drogue, the
reference trajectory becomes a time-varying signal. In this case,
the tracking error needs to be smoothed through a low pass filter.

\textbf{Method 2: Smooth polynomial method.} The reference trajectories
are generated by simple and smooth polynomial functions to meet the
requirement. Reference \cite{fravolini2004modeling} used three-order
polynomials to generate trajectories, while\cite{tandale2006trajectory},
\cite{valasek2017fault} used five-order polynomials. Higher-order
polynomials can produce more complex trajectories, but an appropriate
order is enough to meet the flight requirements. Reference \cite{liu2011flight}
used two-phase polynomial functions to allow the receiver to catch
up with the drogue more quickly in the vertical plane.

\textbf{Method 3:} \textbf{Terminal guidance method.} The idea of
this kind of method stems from the terminal guidance of missiles.
The exact docking of the receiver and the drogue is similar to a missile
hitting a moving target. Reference \cite{ochi2005flight} used Proportional
Navigation Guidance (PNG) and Line-of-Sight (LOS) guidance methods,
which are derived from missile terminal guidance and autonomous landing
of aircraft respectively (It should be noted that reference \cite{ochi2005flight}
mainly explained how to guide the tanker to the designated position
in the tanker coordinate system. Although it was not docking, the
guidance method could be applied to docking). This method calculated
in real time how to adjust the attitude or position of the receiver
in order to approach the desired position optimally. Reference \cite{wang2010verifiable}
used the method of differential games, where the reachable region
of the receiver and the target region determined by the motion of
the drogue are first calculated. Then, the next step is given by calculating
the distance between the two regions. Because the way of giving a
command is similar to terminal guidance, it is also included in Method
3. Additionally, rendezvous guidance laws for tankers and receivers
can also potentially be used for docking \cite{luo2014guidance,tsukerman2018optimal}.
This technique determined in real time how to modify the receiver's
attitude or position in order to get closer to the intended position.

\textbf{Method 4: Intelligent optimization method. }An optimum trajectory
can be provided by intelligent trajectory optimization. In Ref.\cite{liu2020docking},
the gradient of the docking success rate anticipated by a deep neural
network was used to optimize the probe trajectory. Ref.\cite{liu2018deep}
suggested an online trajectory optimization technique to solve the
issue that the bow wave is frequently overlooked in the trajectory
generator. The optimizer created an ideal docking trajectory by using
the bow wave model and the drogue location at the next moment. 

\textbf{Method 5: Preview method.} The preview technique \cite{su2017back,liu2019novel}
is used to address the issue of a slower receiver tracking a faster
drogue which can predict the drogue location to compensate for the
tracking latency. The reference trajectory takes the predicted drogue
motion into account. The preview time is a crucial parameter in deciding
the duration of drogue motion prediction. A fuzzy logic controller
was used in Ref.\cite{su2017back} to choose the preview time. The
controller, however, primarily depends on human intuition, which might
not be optimal. Ref.\cite{liu2019novel} suggested a learning method
that combines deep learning and reinforcement learning to choose the
optimal preview time. 

\subsection{Tracking control}

The main purpose of tracking control is to transform a tracking control
problem into a stabilizing control problem so as to solve the instability
of internal dynamics caused by non-minimum phase property. 

First, the reference state and the reference input are extracted from
the reference trajectory. In the process of converting a tracking
problem into a stabilizing control problem, it is necessary to establish
an state error dynamics first, which can be obtained by subtracting
the reference dynamics from the original dynamics. However, it is
not enough to use only the reference output. Thus, the reference state
and the reference input have to be obtained first. The non-zero setpoint
(NZSP) method proposed in references \cite{valasek2002vision} and
\cite{kimmett2002autonomous} was to derive the reference state and
the reference input based on the reference output and the receiver
model. The NZSP method limits the reference state and the reference
input to be constant, that is, the drogue is stationary relative to
the tanker. In order to overcome this disadvantage, \cite{kimmett2002vision}
improved the NZSP method, and further proposed command generator tracker
(CGT), which could continuously obtain the reference state and input
of the next sampling time with the movement of the drogue. In \cite{tandale2006trajectory}
and \cite{liu2011flight}, Extended State Observers (ESOs) were used
to observe the reference state. Reference \cite{liu2012research}
elaborated on this method in more detail and completeness. The reference
trajectory used in reference \cite{fravolini2004modeling} was a third-order
polynomial of time, and the reference acceleration can be obtained
by twice differential. In the linearized model based on small-angle
approximation, the longitude and lateral acceleration are linearly
related to the angle of attack and sideslip angle. Therefore, the
reference angles of attack and sideslip angles can be obtained directly
by taking the derivative of the trajectory. Then, the ideal system
is established. Based on it, the original system can be converted
into error dynamics by subtracting it. Then, the tracking goal can
be accomplished by stabilizing control for the error dynamics.

Some disturbance rejection terms are also included in tracking control.
The integrator is frequently used to address the problem of tracking
static error caused by atmospheric turbulence, measurement error,
and other random disturbances to a certain extent \cite{valasek2002vision}\cite{kimmett2002vision}\cite{fravolini2004modeling}\cite{dogan2009effects}\cite{wang2008novel}.
Furthermore, robust control in Refs. \cite{murillo2008comparison},\cite{fravolini2003development}
can render the output insensitive to specific noises, primarily atmospheric
turbulence, wind gusts, and sensor-induced noise. Another common method
is to lump all the unknown disturbances, uncertainties, and some higher-order
nonlinearity terms into one disturbance term. Then the lumped disturbance
can be estimated and compensated by a disturbance observer\cite{su2018exact}.
In Ref.\cite{duan2020bionic} a gain-adaptive equivalent sliding mode
control scheme based on the ESO was suggested to attenuate the impact
of wind disturbances and model uncertainties. In Ref.\cite{yu2018wake},
the disturbance observer approach was used to estimate the lumped
uncertainty containing the tanker wake vortex term, and the dynamic
surface control method was chosen to build a position-tracking controller.

AAR is safety-critical, so many scholars are starting to take the
fault-tolerant control issue into account\cite{valasek2017fault,yu2018fault,sevil2015fault}.
Ref.\cite{valasek2017fault} took control-effector failures into account,
and a Structured Adaptive Model Inversion (SAMI) controller was created,
which didn't rely on fault-detection data. In Ref.\cite{yu2018fault},
actuator faults and wake vortices were taken into account as a lumped
uncertainty and estimated by a disturbance observer in order to accomplish
a secure formation flight. Backstepping control was used to accomplish
fault-tolerant control based on the estimated uncertainty. Due to
the near proximity of the tanker and receiver, the collision avoidance
issue was also taken into account in Ref.\cite{chang2019robust},
which transformed it into a state-dependent output-constrained control
problem. Then, the backstepping technique and Barrier Lyapunov Function
(BLF) were used to create a controller. 

It is noteworthy that many studies explicitly design tracking controllers
instead of converting the tracking control problem into a stabilizing
control problem. This is because the non-minimum phase feature of
the receiver is not considered or addressed by other methods. In Refs.\cite{su2018exact,su2018probe},
the six DOF nonlinear receiver model was considered, and backstepping-based
flight controllers were proposed to accomplish high-precision docking
control for AAR in the presence of multiple flow disturbances and
uncertainties. 

The above docking control scheme follows the idea of real-time feedback:
trajectory generating tracking control stabilizing control. Reference
trajectory design and tracking controller design often rely on accurate
models. However, actual models have many uncertainties, such as input
delay and parameter uncertainty. In order to deal with uncertainties,
the feedback control method with an integrator is often adopted. This
method can compensate for uncertain parameters. However, the measurement
delay, the input delay, and the phase lag caused by the integrator
will affect the dynamic performance, control precision, and even stability.
Furthermore, feedback control techniques could cause the receiver
to chase the drogue. The drogue moves quickly in the docking process
due to its light weight. The chasing action may lead to a significant
overshoot and overcontrol for the receiver and may cause impact and
damage to the refueling equipment\cite{ro2011dynamics}. When the
effect does not significantly increase drogue displacement, a rapid
approach technique can be used for docking, but this can easily result
in HWP. According to NATO aerial refueling standards68, this approach
strategy is risky and incorrect. On the other hand, if a sluggish
approach strategy is used, the receiver's bow wave will significantly
shift the drogue. 

Inspired by the pilot training process for manned aircraft refueling,
the docking control can be formulated as a terminal Iterative Learning
Control (ILC) problem or a point-to-point ILC problem\cite{Lin1998Iterative}
\cite{Sun1999Iterative}. The ILC approach is a model-free control
technique that makes use of the repeatability of the considered system
to enhance the system's control performance. If a PDR system's docking
attempt is unsuccessful, the receiver will withdraw to the standby
position in preparation for the next docking attempt, as shown in
Figure \ref{Fig_1.8-3}. That implies that the docking procedure is
repetitive. ILC is a viable option to address the docking control
problem. The basic idea is to use the data from the most recent unsuccessful
docking attempts to change the feedforward input of the docking control.
By doing this, it is possible to compensate for the offset and reject
the repetitive uncertainties. The problem of the slower dynamic receiver
tracking the quicker dynamic drogue may be resolved by ILC. Another
advantage of ILC is that it allows identification performed after
one docking trial. Compared with real-time identification at each
sensor sampling period, the identification here can use more sampling
data and has sufficient calculation time. This can reduce the requirement
of visual sensor and improve its feasibility. The ILC methods suitable
for the probe-and-drogue AAR is different from the traditional ILC
in that it cannot guarantee the actual initial position of the receiver
is the same as the expected initial value. What is more, the task
needs to make full use of model information to reduce the iterative
number. The aerial refueling docking controllers in Refs.\cite{dai2018terminal,ren2019reliable,jinrui2020docking}
were designed using iterative learning, which successfully averted
the drawbacks of slower dynamics tracking faster dynamics and overcontrol
in feedback control. These ILC techniques used an adjoint operator
to solve the non-minimum phase problem. Iterative learning control
can effectively handle repetitive disturbances but cannot suppress
non-repetitive disturbances. The ability to suppress sudden gusts
or random turbulences is poor during aerial refueling docking. 
\begin{figure}
\begin{centering}
\includegraphics[width=0.6\textwidth]{Figures/Figs_Ch1/Fig_1\lyxdot 8\lyxdot 3}
\par\end{centering}
\caption{ILC-based docking operation in AAR.\cite{jinrui2020docking}}

\centering{}\label{Fig_1.8-3}
\end{figure}


\subsection{Stabilizing Control}

To suppress uncertainties and disturbances and enable the error dynamics
state to converge to zero, stabilizing control is used. According
to the design methods of the traditional flight control system, classical
state feedback and Proportional-Integral-Derivative (PID) method can
be used to control the system. Reference \cite{ochi2005flight} applied
state feedback to stabilize the system. Reference \cite{li2010uav}
used PID method to design for each single-input-single-output channel
for the inner and outer loops. Its outer loop PID coefficients are
adjusted adaptively by particle swarm optimization. The modified PID
method can restrain the vortex disturbance caused by the tanker wingtip.
In contrast, most methods prefer the Linear Quadratic Regulation (LQR)
method \cite{valasek2002vision} \cite{kimmett2002vision} \cite{tandale2006trajectory}
\cite{fravolini2004modeling} \cite{rehan2012robust} \cite{wang2011precise}.
Linear quadratic optimization is a convex optimization with a unique
extremum, so this method is often used to design the optimal feedback
gain matrix. Some other methods to suppress perturbation and nonlinearity
also use quadratic optimization, such as the adaptive dynamic inverse
method in \cite{gai2012trajectory} and the L1 adaptive method in
\cite{wang2010verifiable}, \cite{wang2008novel}. Reference \cite{stepanyan2004aerial}
used the adaptive control to stabilize the controller and track an
adaptive signal. In reference \cite{kimmett2002autonomous}, Linear
Quadratic Gaussian (LQG) regulator was used to reduce the effect of
Gaussian white noise caused by measuring sensors in the system so
as to improve the stabilization effect. Although the actual aircraft
model is nonlinear, the stabilizing controller design is generally
built on linear models. Consequently, when the disturbance is significant,
the control performance may suffer. Reference \cite{elliott2010investigating}
proposed feedback linearization to solve this problem. Few papers
are based on the six degrees of freedom (DOF) nonlinear receiver model.
For such a purpose, backstepping based flight controller was proposed
to solve the AAR docking control problem with high precision in the
presence of multiple flow disturbances and uncertainties \cite{su2018exact}\cite{su2018probe}. 

\subsection{Station-keeping control}

After a successful capture, the receiver and tanker must maintain
relatively stationary during the refueling phase to transmit the fuel.
In other words, the receiver must hold a specific location under the
coordinate frame of the tanker without significantly deviating from
it. For two main reasons, station-keeping control for AAR is a challenging
task. The first reason is that the receiver is disturbed by atmospheric
disturbances and the tanker's wake vortex. The second reason is that
the receiver's mass, inertia, and center of mass will alter due to
the fuel transfer during the refueling phase.

In order to maintain the location, station-keeping control must first
establish the reference state, which transforms the issue into a stabilizing
control problem. Generally speaking, by fixing the reference frame
on the tanker, reference state can be regarded as a constant value.
The station-keeping control method is basically consistent with the
stabilizing control mentioned above. For example, the LQR method was
adopted in references \cite{dogan2009effects}, \cite{elliott2009improving}
and \cite{lee2013estimation}, the L1 adaptive method was adopted
in reference \cite{wang2010verifiable}, and the PID method was adopted
in \cite{ross2006autonomous}. In view of the receiver mass change
in the process of fuel transferring, the control method of adaptive
gain (gain scheduling) was adopted in references \cite{wang2010verifiable},
\cite{dogan2009effects}. These are based on rigorous and accurate
modeling, such as the result in \cite{dogan2009effects}, which is
based on the variable mass modeling of the receiver under wake vortex
disturbance in \cite{Waishek2009Derivation}. Reference \cite{kriel2013receptacle}
proposed that the controlled plant should be the position of the probe
rather than the position of the center of gravity of the receiver.
Therefore, the relative position of the center of gravity of the receiver
and its probe was used to modify the original model, and the controller
was then designed. In \cite{an2018relative}, double power reaching
law based sliding mode controller was designed to control the receiver
translational motion relative to the tanker aircraft in the outer
loop while Active Disturbance Rejection Control (ADRC) technique was
applied to the inner loop to stabilize the receiver. 

Prior information about the refueling system or disturbances can also
be used to solve some problems encountered in station keeping. Reference
\cite{elliott2009improving} dealt with the problem of how to control
a receiver when the flight condition of the tanker changes. This paper
considered that the position and attitude information of the tanker
could be obtained by communication as the feedforward, which was then
incorporated into the LQR. The robust design method was consistent
with the stabilizing control problem in solving this problem, i.e.,
to minimize the gain of the disturbance or disturbance on the output
as much as possible \cite{rehan2012robust}. Reference \cite{sun2013uav}
adopted the ADRC design method to design the station-keeping controller.
In \cite{pachter1997design}, Quantitative Feedback Theory (QFT) was
used to design controllers to guide the receiver and tanker formation.
This control method was also applied to station keeping \cite{guo2010design}.

In the past ten years, station-keeping control has drawn some interest.
Ref.\cite{an2018relative} examined the effects of fuel injection
in terms of the receiver's mass change and center-of-gravity change.
An inner and outer loop controller structure was considered. The sliding
mode controller was developed in the outer loop to regulate the relative
motion between the receiver and the tanker, and the ADRC method was
used in the inner loop to stabilize the receiver. A station-keeping
control based on additive-state-decomposition was suggested in Ref.\cite{ren2021additive}
to take the receiver's nonlinearity into account. With the aid of
the additive state decomposition, the effects of nonlinearity and
uncertainties were decomposed, making the remaining control design
simpler. The predefined-time, finite-time, and fixed-time attitude
stabilization controls for receiver aircraft were investigated in
Refs.\cite{wu2021dynamic,wu2021predefined,wu2022predefined} in order
to maintain the intended attitude of the receiver for a long period
of time to receive fuel while achieving a quicker convergence rate.
In Ref.\cite{wu2021predefined} , a sliding manifold was created to
enable the sliding mode phase achieved within the predetermined time,
and after that, a robust stabilization controller was used to accomplish
the predefined-time attitude stabilization. The benefit of the method
is that the system settling time can be selected arbitrarily and independent
of system states. A nonsingular sliding mode based adaptive controller
was created in Ref.\cite{wu2021dynamic} to accomplish the faster
finite-time stability of the closed-loop system while considering
the receiver model's uncertain and time-varying inertia, wind disturbances,
and change in center of mass. However, the receiver models used in
Refs.\cite{wu2021dynamic,wu2021predefined,wu2022predefined} are merely
six-degree-of-freedom models with the force and torque inputs, and
the aerodynamic models of force and torque are not considered. Additionally,
the chattering of the sliding mode control is still a problem.

\subsection{Anti-HWP control}

The refueling pod is frequently outfitted with a reel system, specifically
an HDU\cite{vassberg2003numerical}, in order to maintain the hose
tension steady and prevent the hose whipping phenomenon\cite{ro2011dynamics}.
The hose whipping phenomenon brought on by the receiver's extreme
closing speed needs to be repressed, so the HDU control by reeling
in/out the hose is required for safety reasons. By adjusting the hose
length, HDU can prevent the development of extra slack in the hose
and maintain the internal hose tension. The conventional way to rewind
the hose is to outfit the refueling pod with a tensator, a spring-loaded
take-up mechanism. However, the approach is a passive method in nature,
and it is challenging to reconstruct the tensator's mechanism. Additionally,
because the reel take-up speed is slower than the closing speed, the
hose whipping phenomenon is adversely affected. As a result, as seen
in Figure \ref{Fig_1.8-4}, many scholars have begun to investigate
using a Permanent Magnet Synchronous Motor (PMSM) to drive the HDU.
Through a reducer, the PMSM directs the HDU to release and retrieve
the hose. Ref.\cite{wang2014dynamic} converted the hose length control
to the angular control of the PMSM, and then suggested an active control
strategy based on an integral sliding mode backstepping controller
design to prevent the hose whipping phenomenon. However, the suggested
backstepping control's computation of higher-order command derivatives
will result in a problem with an exponential explosion. A command-filtered
backstepping sliding mode controller was suggested in Ref.\cite{he2017command}
as a remedy for this problem. It was suggested in Ref.\cite{dai2019hose}
to use two different kinds of HDU controllers to regulate the hose
length to stabilize the drogue movement and prevent the hose whipping
phenomenon.

\begin{figure}
\begin{centering}
\includegraphics[width=0.6\textwidth]{Figures/Figs_Ch1/Fig_1\lyxdot 8\lyxdot 4}
\par\end{centering}
\caption{Schematic of Hose-Drum Unit\cite{wang2014dynamic}}

\centering{}\label{Fig_1.8-4}
\end{figure}


\subsection{Drogue stabilization and control}

Some studies concentrated on boosting the damping of drogue motion
to slow down the drogue dynamics in order to address the issue of
a slower dynamic receiver tracking a fast-moving drogue. These studies
included the addition of an active stability control device. The drogue
motion can be indirectly controlled by the HDU, which is frequently
regarded as a passive control technique. The main topic of this part
is the active control using the drogue's built-in control surfaces.
Ref.\cite{pud-drogue} tended to create a drogue with an active controller
and self-stabilization. The design aimed to increase the stability
of the drogue and make it difficult to be disturbed\cite{pud-drogue}
\cite{kuk2013design}. The fundamental design of the autonomous drogue
is the installation of four control surfaces at the point where the
drogue connects to the hose so that the drogue can adjust its location.
A self-stabilized drogue's design arrangement, as shown in Figure
\ref{Fig_1.8-5}, was provided in Ref.\cite{kuk2013design}, and its
performance was assessed in a wind tunnel. Ref.\cite{yuan2017study}
also took into account a similar self-stabilized drogue arrangement.
Ref.\cite{AAR-2014} lists some additional self-stabilized drogue
designs. The dynamic response of the integrated system, which consists
of the hose, the drogue, and the control surfaces, was examined in
Ref.\cite{garcia2018dynamic} by considering the control surfaces
placed in the drogue. A fractional-order controller was developed
in Ref.\cite{sun2018fractional} to actively regulate the drogue in
order to maintain its location within a range. An improved pigeon-inspired
optimization technique was suggested in response to the challenge
of selecting the controller parameter of the fractional-order controller.
Ref.\cite{sun2022active} developed active disturbance rejection controllers
to improve the drogue's anti-disturbance capabilities. 

To increase the control precision during aerial refueling docking,
many researchers have studied the vibration control of the hose and
drogue assembly using the control surfaces installed on the drogue.
By using the backstepping technique to reduce the elastic vibration
of the flexible hose, Ref.\cite{liu2020vibration} developed boundary
control. Ref.\cite{chang2019adaptive} developed a multi-objective
adaptive controller to simultaneously address the input nonlinearities
of dead-zone, input constraints, and partial state constraints. Additionally,
the fault-tolerant control for the refueling hose and drogue assembly
with regard to actuator failure was also researched\cite{liu2021adaptive,liu2022adaptive},
similar to the fault-tolerant control for the receiver. For a refueling
hose with variable length and constrained output, Ref.\cite{liu2021adaptive}
evaluated the partial effectiveness loss of the actuator and suggested
an adaptive barrier-based fault-tolerant control. A control strategy
that combines adaptive methods and redundancy actuators was used in
Ref.\cite{liu2022adaptive} to address the possibility of a partial
or even complete loss of the actuator's effectiveness. 

\begin{figure}
\begin{centering}
\includegraphics[width=0.6\textwidth]{Figures/Figs_Ch1/Fig_1\lyxdot 8\lyxdot 5}
\par\end{centering}
\caption{Schematic of Hose-Drum Unit\cite{liu2022adaptive}}

\centering{}\label{Fig_1.8-5}
\end{figure}

The active drogue control may offer a fresh approach to the docking
control issue. In this situation, a novel control strategy that controls
the drogue to connect with the probe could be suggested in place of
controlling the probe to dock with the drogue. This method can solve
the issue of a fast-moving drogue being tracked by a sluggish dynamic
receiver. It is possible to reduce overcontrol and prevent chasing
actions while increasing docking safety. There are still some issues,
though: If the control surfaces are too big, the interference that
they experience may result in significant drogue drift; if the control
surfaces are too tiny, they lack sufficient controllability. Furthermore,
a power supply for the self-stabilizing drogue is required because
it is far from the tanker. Electricity leakage and fire safety must
be considered if electronic control is used.

The control for PDR has been examined up to this point, and Table
\ref{Tab_1.2} displays the methods taken and the problems addressed
in some research and scholarly papers. It is noteworthy that designs
for both flying-boom aerial refueling and probe-and-drogue aerial
refueling are included, with the former's introduction meant to provide
some related techniques. Section \ref{sec:Characteristics-and-problems}
provides a description of the docking characteristics and requirements
in the table. Table \ref{Tab_1.2}'s symbol \textquotedblleft \Checkmark \textquotedblright{}
denotes that the method took into account the corresponding requirements
and characteristics, while the vacant space denotes the other way
around. In Table \ref{Tab_1.2}, U1, U2, U3,U4 mean atmospheric disturbance,
bow wave, time-varying mass and inertia, and tanker wake vortex, respectively;
R1, R2, R3 denote high precision, high safety, and high efficiency,
respectively.

\begin{landscape} 

\begin{longtable}[c]{c|>{\centering}m{0.025\columnwidth}|>{\centering}m{0.025\columnwidth}|>{\centering}m{0.025\columnwidth}|>{\centering}m{0.03\columnwidth}|>{\raggedright}p{0.025\columnwidth}|>{\raggedright}p{0.025\columnwidth}|>{\raggedright}p{0.025\columnwidth}|>{\centering}p{0.03\columnwidth}|>{\raggedright}p{0.025\columnwidth}|>{\centering}m{0.08\columnwidth}|>{\centering}m{0.08\columnwidth}|>{\centering}m{0.06\columnwidth}|>{\centering}p{0.08\columnwidth}|>{\centering}p{0.08\columnwidth}|>{\centering}p{0.08\columnwidth}}
\caption{Summary of control problems and methods during autonomous aerial docking}
 \tabularnewline
\hline 
\multirow{3}{*}{Reference} & \multicolumn{6}{c|}{Characteristic} & \multicolumn{3}{c|}{Requirement} & \multicolumn{6}{c}{Controller}\tabularnewline
\cline{2-16} \cline{3-16} \cline{4-16} \cline{5-16} \cline{6-16} \cline{7-16} \cline{8-16} \cline{9-16} \cline{10-16} \cline{11-16} \cline{12-16} \cline{13-16} \cline{14-16} \cline{15-16} \cline{16-16} 
 & \multicolumn{4}{c|}{Uncertainty} & \multirow{2}{0.025\columnwidth}{\centering{}Slow dyn.} & \multirow{2}{0.025\columnwidth}{\centering{}NMP} & \multirow{2}{0.025\columnwidth}{\centering{}R1} & \multirow{2}{0.03\columnwidth}{\centering{}R2} & \multirow{2}{0.025\columnwidth}{\centering{}R3} & \multicolumn{3}{c|}{Docking controller} & \multirow{2}{0.08\columnwidth}{Station-keeping controller} & \multirow{2}{0.08\columnwidth}{Anti-HWP control } & \multirow{2}{0.08\columnwidth}{Drogue stabilization and control }\tabularnewline
\cline{2-5} \cline{3-5} \cline{4-5} \cline{5-5} \cline{11-13} \cline{12-13} \cline{13-13} 
 & U1 & U2 & U3 & \centering{}U4 &  &  &  &  &  & Command 

generator  & Tracking 

controller  & Sta. 

controller  &  &  & \tabularnewline
\hline 
\cite{Dibley-2007-2} & \Checkmark{} & \Checkmark{} & \Checkmark{} &  & \Checkmark{} & \Checkmark{} & \Checkmark{} & \Checkmark{} &  &  &  &  &  &  & \tabularnewline
\hline 
\cite{valasek2002vision} & \Checkmark{} &  &  &  &  & \Checkmark{} & \Checkmark{} &  & \Checkmark{} & Step Signal + PI & NZSP & LQR &  &  & \tabularnewline
\hline 
\cite{kimmett2002vision} & \Checkmark{} &  &  &  &  & \Checkmark{} & \Checkmark{} &  & \Checkmark{} & Time-varying signal + PI & CGT & LQR &  &  & \tabularnewline
\hline 
\cite{kimmett2002autonomous} & \Checkmark{} &  &  &  &  & \Checkmark{} & \Checkmark{} &  & \Checkmark{} & Step signal + PI & NZSP & LQG &  &  & \tabularnewline
\hline 
\cite{tandale2006trajectory} & \Checkmark{} &  &  &  &  & \Checkmark{} & \Checkmark{} &  & \Checkmark{} & 5th-order polynomial & Extended state observer & LQR &  &  & \tabularnewline
\hline 
\cite{fravolini2004modeling} & \Checkmark{} &  &  &  &  & \Checkmark{} & \Checkmark{} &  & \Checkmark{} & 3th-order polynomial & LQR + integrator & LQR &  &  & \tabularnewline
\hline 
\cite{stepanyan2004aerial} & \Checkmark{} & \Checkmark{} &  &  &  & \Checkmark{} & \Checkmark{} &  &  & Nonlinear reference model & Differential games & Adaptive control &  &  & \tabularnewline
\hline 
\cite{wang2010verifiable} & \Checkmark{} &  &  &  &  & \Checkmark{} & \Checkmark{} &  & \Checkmark{} & Time-varying signal + LPF & LQR + Integrator & L1 Adaptive &  &  & \tabularnewline
\hline 
\cite{wang2009l1} & \Checkmark{} &  & \Checkmark{} &  &  & \Checkmark{} & \Checkmark{} &  & \Checkmark{} &  &  &  & Gain scheduling + L1 adaptive &  & \tabularnewline
\hline 
\cite{ross2006autonomous} &  &  &  &  &  & \Checkmark{} &  &  &  &  &  &  & Gain scheduling + PID &  & \tabularnewline
\hline 
\cite{ochi2005flight} &  &  &  &  &  & \Checkmark{} & \Checkmark{} &  &  & PNG + LOS & Integrator + State feedback & State feedback &  &  & \tabularnewline
\hline 
\cite{dogan2009effects} & \Checkmark{} &  & \Checkmark{} &  &  & \Checkmark{} & \Checkmark{} &  & \Checkmark{} &  &  &  & Gain scheduling + LQR &  & \tabularnewline
\hline 
\cite{pachter1997design} & \Checkmark{} &  &  &  &  & \Checkmark{} & \Checkmark{} &  &  &  &  &  & QFT &  & \tabularnewline
\hline 
\cite{murillo2008comparison} & \Checkmark{} &  &  &  &  & \Checkmark{} & \Checkmark{} &  & \Checkmark{} &  & 1. Robust servomechanism design 2. Model following design 3. H$\infty$ & 1. LQR 2. LQR 3. H$\infty$ &  &  & \tabularnewline
\hline 
\cite{fravolini2003development} &  & \Checkmark{} &  &  &  & \Checkmark{} & \Checkmark{} &  &  & Tracking error + PI & H$\infty$ & H$\infty$ &  &  & \tabularnewline
\hline 
\cite{elliott2009improving} &  &  &  &  &  & \Checkmark{} & \Checkmark{} &  & \Checkmark{} &  &  &  & Feedforward + LQR &  & \tabularnewline
\hline 
\cite{rehan2012robust} & \Checkmark{} &  & \Checkmark{} &  &  & \Checkmark{} & \Checkmark{} &  &  &  &  &  & PID + H$\infty$ &  & \tabularnewline
\hline 
\cite{lee2013estimation} & \Checkmark{} &  &  &  &  & \Checkmark{} & \Checkmark{} &  & \Checkmark{} &  &  &  & LQR + Integrator &  & \tabularnewline
\hline 
\cite{sun2013uav} & \Checkmark{} &  & \Checkmark{} &  &  & \Checkmark{} & \Checkmark{} &  &  &  &  &  & ADRC &  & \tabularnewline
\hline 
\cite{liu2011flight} & \Checkmark{} & \Checkmark{} &  &  &  & \Checkmark{} & \Checkmark{} &  & \Checkmark{} & Two-phase polynomials & Extended state observer & LQR &  &  & \tabularnewline
\hline 
\cite{li2010uav} & \Checkmark{} &  &  &  &  & \Checkmark{} & \Checkmark{} &  &  &  & PID & PID &  &  & \tabularnewline
\hline 
\cite{an2018relative} & \Checkmark{} &  & \Checkmark{} & \Checkmark{} &  &  &  &  &  &  &  &  & SMC+ ADRC  &  & \tabularnewline
\hline 
\cite{wu2021dynamic} & \Checkmark{} &  & \Checkmark{} & \Checkmark{} &  &  &  &  &  &  &  &  & Sliding mode-based adaptive control  &  & \tabularnewline
\hline 
\cite{su2018exact} & \Checkmark{} &  &  &  & \Checkmark{} &  & \Checkmark{} &  &  &  & Backstepping higher order SMC  &  &  &  & \tabularnewline
\hline 
\cite{liu2020vibration} &  &  &  &  &  &  &  & \Checkmark{} &  &  &  &  &  &  & Backstepping 

control\tabularnewline
\hline 
\cite{liu2018deep} & \Checkmark{} & \Checkmark{} &  & \Checkmark{} &  &  &  &  &  & Deep learning + Reference observer &  &  &  &  & \tabularnewline
\hline 
\cite{valasek2017fault} &  &  &  &  &  &  &  & \Checkmark{} &  & Fifth-order polynomial & Fault-tolerant SAMI control &  &  &  & \tabularnewline
\hline 
\cite{liu2019novel} & \Checkmark{} &  &  & \Checkmark{} & \Checkmark{} &  & \Checkmark{} &  &  & Learning-based preview method & LQR+HOSM controller  &  &  &  & \tabularnewline
\hline 
\cite{ren2019reliable} & \Checkmark{} & \Checkmark{} &  &  & \Checkmark{} & \Checkmark{} &  & \Checkmark{} &  &  & TILC &  &  &  & \tabularnewline
\hline 
\cite{he2017command} & \Checkmark{} &  &  & \Checkmark{} &  &  &  & \Checkmark{} &  &  &  &  &  & Command filtered backstepping SMC & \tabularnewline
\hline 
\cite{sun2022active} & \Checkmark{} & \Checkmark{} &  & \Checkmark{} & \Checkmark{} &  & \Checkmark{} & \Checkmark{} & \Checkmark{} &  &  &  &  &  & ADRC\tabularnewline
\hline 
\end{longtable}
\begin{center}
\label{Tab_1.2}
\par\end{center}

\end{landscape}

\section{Progress in Safety of PDR}

Docking safety is a crucial problem in the refueling phase of PDR.
With the gradual maturity of the modeling and control technology of
PDR, many researchers started to focus on the safety problem.

\subsection{Safety Analysis}

In practice, the docking safety analysis has important guiding significance
for the docking maneuver decision of AAR. 

\subsubsection{Docking success rate evaluation}

Currently, most control techniques described in academic and research
papers demonstrate accurate docking ability under specific circumstances.
However, in the real world, it is extremely challenging to accomplish
exact docking every time. In NASA's trial report \cite{Dibley-2007-2},
only three of the six docking flight attempts were successful due
to various disturbances. According to Ref.\cite{wang2017approach},
the success rate for docking during manned aerial refueling is approximately
35\%, while the success rate for docking during NASA's UAV aerial
refueling is approximately 60\%. This suggests that the docking success
rate will be reduced under challenging circumstances. The docking
success rate under disturbances must therefore be taken into account.
There is, however, little study on quantitative modeling and examination
of the AAR docking success rate. Online and offline evaluations of
the docking success rate are available. How to acquire a docking envelope
is the subject of the offline assessment. The docking success probability
can be predicted using the envelope obtained offline and the receiver's
present location and velocity. The Monte Carlo method, a safety evaluation
technique for uncertain systems, is a straightforward approach. However,
performing numerous Monte Carlo simulations, which takes time, is
necessary for a decent assessment outcome. A technique using the likelihood
of the drogue center situated in the capture area was suggested to
determine the online docking success probability by considering the
drogue motions under atmospheric disturbances\cite{wang2017approach}.
The reachability analysis technique, used by Ref.\cite{liu2021stochastic},
is another theory approach. In Ref.\cite{liu2021stochastic}, the
probability of the receiver joining the target set within a specified
time period was calculated using the Markov chain stochastic approximation
method, which required much less simulation time than the Monte Carlo
method. However, the reachability analysis method's memory and computation
needs increase rapidly with dimension. Online computation will be
difficult if many parameters are considered or the relative motion
model has a large dimension. Additionally, Ref.\cite{liu2020docking}
suggested a real-time safety evaluation network, a neural network
based on deep learning, to predict the docking success rate based
on the current docking state, preventing the difficult reachability
analysis. There was also established a safety margin between the probe
and the drogue. But getting lots of training data is a challenge. 

\subsubsection{Docking reachability calculation}

The problem \textquotedbl Is there a controller that satisfies the
docking criteria given the uncertainty model and the constraint on
the docking success rate?\textquotedbl{} is the goal of this part.
This problem pertains to the calculation of the docking reachable
set A that correlates to the docking target set B, as shown in Figure
\ref{Fig_1.8-6}, and can be seen as an inverse problem of the assessment
of the docking success rate. The conventional controllability problem
and this one are related. However, it does not examine whether the
receiver state can be controlled to the origin, as is the case with
conventional controllability, but rather whether the drogue can be
caught to a set with the proper relative docking velocities, etc.
Another distinction is that docking reachability is a form of probabilistic
reachability due to random perturbations and uncertainties. Additionally,
certain essential safety requirements will also limit the docking
capability. Ref.\cite{wang2019towards} developed a docking reachability
calculation algorithm based on the Hamilton-Jacobi equation and an
algorithm for real-time docking success probability estimation regarding
the relative distance between the drogue and the probe. However, the
drogue position was only possible for a two-dimensional normal distribution
in the study. On this basis, we should further define the degree of
reachability, as shown in Figure \ref{Fig_1.9}. If there is no reachability,
then we need to abandon the docking. What is more, the degree of reachability
can, in turn, propose a demand for the atmospheric environment and
the receiver. Aerial refueling docking may not be carried out under
any receiver and any docking environment. Many factors may result
in that no controller meets the docking requirements, such as the
tanker size, atmospheric environment, algorithm processing time and
effect, sensor dynamic response and accuracy, control surface dynamic
response and accuracy. 

\begin{figure}
\begin{centering}
\includegraphics[width=0.6\textwidth]{Figures/Figs_Ch1/Fig_1\lyxdot 8\lyxdot 6}
\par\end{centering}
\caption{Airspace division for aerial refueling}

\centering{}\label{Fig_1.8-6}
\end{figure}

\begin{figure}
\begin{centering}
\includegraphics[width=0.5\textwidth]{Figures/Figs_Ch1/Fig_1\lyxdot 9}
\par\end{centering}
\caption{Degree of reachability}

\centering{}\label{Fig_1.9}
\end{figure}

On the other hand, most research and academic publications consider
the problem of aerial docking as an accurate tracking problem and
rarely consider the damage of the drogue and the potential danger
in the flight. In Figure \ref{Fig_1.10}, for example, when the probe
passes through the drogue, the docking will be continued by following
the idea of tracking control. However, to continue the mandatory docking
at this point, a larger pitch command is required, which is prone
to a heavy collision between the probe and the drogue. Even without
collision, large commands can cause large overshooting and make the
drogue damaged, as shown in the video \cite{Vedio1}. In order to
ensure docking safety, it is necessary to set impassable or unsafe
areas (as shown in Figure 13) and allow the receiver to abandon the
docking attempt once entering these areas. As shown in Figure \ref{Fig_1.11},
Ref. \cite{Dibley-2007-2} designated the \textquotedbl capture\textquotedbl{}
area and the \textquotedbl miss\textquotedbl{} area (the docking
task declared to failed when entering this area) by segmenting the
area around the drogue into different areas. When the probe enters
the \textquotedbl miss\textquotedbl{} area, the docking attempt will
be immediately aborted. However, this study did not offer a theory
or method for the area partition. In Ref.\cite{ding2012reachability},
the authors divided the flight safety region for aerial refueling
using the controllable region to constrain the flight trajectory.
Reference \cite{Ro-2010-5} referred to the need for an appropriate
capture speed to open the fuel valve without causing hose whip. According
to some studies, the controller's command should fall within a reasonable
range. The aerial refueling accident shown in the video \cite{Vedio2}
is caused by unreasonable overshooting. However, the majority of these
academic and research publications only provide ad-hoc guidelines
based on experience, which calls for additional research.

\begin{figure}
\begin{centering}
\includegraphics[width=0.5\textwidth]{Figures/Figs_Ch1/Fig_1\lyxdot 10}
\par\end{centering}
\caption{A kind of unsafe relative position}

\centering{}\label{Fig_1.10}
\end{figure}

\begin{figure}
\begin{centering}
\includegraphics[width=0.5\textwidth]{Figures/Figs_Ch1/Fig_1\lyxdot 11}
\par\end{centering}
\caption{Docking area division \cite{Dibley-2007-2}}

\centering{}\label{Fig_1.11}
\end{figure}


\subsubsection{Optimal flight condition determination}

This part pays attention to \textquotedbl At what flight condition
(altitude and cruise speed) the docking control has the highest success
rate?\textquotedbl . Here, the term \textquotedbl optimal flight
condition\textquotedbl{} refers to the altitude and cruise speed that
will result in the greatest likelihood of docking success. Different
docking altitudes and speeds will produce various receiver models
at the equilibrium point, which will then result in various reachability
probabilities. Therefore, better docking altitude and speed for the
same controller can increase the docking success rate. The solution
to the problem is of significance for aerial refueling docking. Ref.\cite{liu2021reachability}
used the reachability analysis to create an optimization problem whose
solution is the optimal trim state. The altitude and speed that correspond
to the largest volume of the reachable set are the optimal trim state.

\subsection{Safety-oriented decision}

In terms of \textquotedbl How to design a safe flight decision for
functional failures or mission failures?\textquotedbl , only limited
studies can be found, or just very simple results are given on the
safety-oriented decision-marking. In reality, various external factors,
such as excessive turbulence that lasts too long to stabilize the
aircraft in a particular position or a malfunctioning receiver's navigation
system that prevents the drogue from being positioned, can cause the
AAR mission to fail. Making safe and trustworthy decision under these
unusual circumstances is necessary to guarantee a safe flight. It
is necessary to think about how to combine the requirements into a
final reliable flight decision-making scheme in addition to the proposed
safety requirements. A straightforward safety-oriented docking strategy
with six docking modes was proposed in \cite{liu2020docking} as a solution to this
issue. It was based on the safety margin determined using the relative
distance between the probe and the drogue and the docking success
rate. In order to make decisions for AAR, Ref.\cite{dong2019failsafe}
proposed a failsafe mechanism based on supervisory control of state
tree structures. According to exterior directives and aircraft flight
states, the complete failsafe system can direct the receiver's following
actions.

\section{Chapter Summary and possible future work}

\subsection{Chapter Summary}

Since the middle of the 20th century, scholars have been interested
in PDR because of its huge military and civilian importance. The refueling
phase is of particular concern. There are a lot of uncertainties during
the PDR's refueling phase, which affect the receiver's motion, the
drogue's movement, the receiver model, and the docking's initial state.
The receiver also exhibits slower dynamic and non-minimum phase characteristics.
High efficiency, high safety, and high precision technical requirements
must be met during the refueling phase. 

In the refueling phase of probe-and-drogue autonomous aerial refueling,
the most recent advancements and fresh research findings in modeling,
control, and safety are compiled in this chapter. The study on modeling,
which includes modeling of aircraft, refueling equipment, and wind
disturbances, is first given. Due to the hose's flexibility, it should
be noted that modeling the refueling equipment is fairly difficult.
Then, the control study is divided into six sections: command generator,
tracking control, stabilizing control, station-keeping control, anti-HWP
control, and drogue stabilization and control. Finally, safety-related
technologies are presented, such as safety analysis and safety-focused
decision-making, which merit further investigation. 

\subsection{Possible Future Work }

Despite all the advancements previously mentioned, numerous issues
still need to be researched, and numerous technologies still have
room for improvement. Future advancements and research generally tend
to make refueling docking safer, more accurate, and more efficient
in accordance with the requirements in the refueling phase listed
in the Introduction.

\subsubsection{Possible Future Work for high-safety requirement}

For the high-safety requirement, further research could be done in
the areas of intelligent autonomous mechanical design, safety assessment,
modular controller design, dependable decision-making, and adaptive
vision navigation.

\textbf{(1) Intelligent autonomous mechanical design }

The fact that the hose is flexible and the motion of the drogue is
rapid and passive is the main cause of PDR's difficulty in docking.
Thus, a better docking mechanism could be exploded. Currently, drogues
with automated control surfaces installed are under consideration.
The drogue's movement can be stabilized and controlled by the control
surfaces. In this situation, a new control scheme can be suggested
where the drogue is controlled to link up with the probe instead of
controlling the probe to dock with the drogue. With this plan, the
issue of a fast-moving drogue being tracked by a slower dynamic receiver
can be solved. It is possible to avoid chasing action and overcontrol,
and docking safety can be increased. 

\textbf{(2) Safety assessment} 

Safety assessment is a further topic that needs to be researched because
high safety is the most important requirement for AAR. The safety
analysis of PDR has been the subject of some studies thus far, but
more consideration and work are still required. The docking success
rate assessment, docking reachability calculation, division of safe
and unsafe areas, determination of the optimal flight condition, robust
margin analysis, etc., are all included in the safety assessment.
Deep learning-based methods are a viable option for fitting complex
unknown relationships and can be used in safety assessment. However,
the mathematical basis for safety analysis must be uncovered because
deep learning is somewhat unexplainable. The reachability analysis
method is a preferred theoretical approach, in which probabilistic
reachability aids in carrying out the safety analysis under stochastic
disturbances. 

\textbf{(3) Modular controller design }

The modular controller architecture is advantageous for future controller
updates and alterations. The control components are linked hierarchically
and appropriately to form the overall controller. If the modular controller
architecture is used, it will be possible to modify the guidance module
while maintaining the low-level control module's original configuration
for various tasks. This makes the transition to a new controller simpler.
Reusability is essential for safety-critical applications like PDR
because many actual flight tests have demonstrated the low-level control
module's dependability. Additionally, it is not suggested to access
the throttle, elevator, aileron, and rudder directly for safety reasons.
It is also time and money efficient for control components to be reused
in various applications. Additionally, more investigation into a low-level
controller that uses velocity as the input merits attention. The receiver's
position control is frequently the target of the velocity controller.
But the velocity controller designed here is required to aim at the
position control of the probe tip in relation to the receiver's attitude
for docking. This presents many difficulties. 

\textbf{(4) Dependable decision-making} 

In order to control and manage aerial refueling, a reliable decision-making
scheme must be developed. On the safety-oriented decision-marking,
however, there are few studies available or the findings are very
straightforward. The refueling operation can be affected by a variety
of factors, such as when the turbulence is too intense for an extended
period of time to stabilize the aircraft in a specific position, when
the actuator malfunctions for unknown reasons and makes it difficult
to control the attitude and position, when an enemy finds the aircraft
while it is being refueled, and when the receiver's navigation system
is malfunctioning and prevents the drogue from being positioned. Due
to these various factors, the aircraft should be able to operate in
numerous modes, and multi-mode decision-making is important for regulating
the aircraft to ensure safety. One option is to use the Finite State
Machine (FSM) technique, which calls for thorough requirements analysis,
mode specification, and event definition. 

\textbf{(5) Adaptive vision navigation }

During the rendezvous and joining phases of AAR, global positioning
and wireless communication technologies frequently provide relative
position information. They might, however, be disturbed. In contrast,
vision-based navigation devices offer the relative distance and speed
between the probe and the drogue during the refueling phase. The system
can work fully autonomously, but the observation distance and the
field of vision are somewhat at odds with one another, which is a
crucial restriction for the system. On the other hand, short focal
lenses have a broad field of vision but a close observation distance.
On the other hand, telephoto lenses have a narrow field of vision
but a long observation distance. As a result, adaptive vision navigation
technology is anticipated to achieve both a broad field of view and
a long observation distance. The rendezvous, joining, and refueling
phases can all use the adaptive vision navigation system. Installing
a zoom lens or numerous lenses with various focal lengths as compound
eyes is one option. The related algorithms with these new lenses are
rife with difficulty to obtain accurate and robust relative poses
in all-weather flight. 

\subsubsection{Possible Future Work for high-precision requirement }

For the high-precision requirement, online learning of complex disturbances
and image-based visual servo control need more attention. 

\textbf{(1) Online learning of complex disturbances }

The refueling phase is characterized by complicated disturbances.
The full consideration of disturbances is important for high-precision
control. To deal with disturbances, PDR primarily uses two types of
control designs. The first is to model disturbances, followed by adding
the disturbance model to the aircraft model. The ultimate control-oriented
model is then created using the combined model. The second method
involves combining all disturbances and unmodeled dynamics into a
single disturbance that is estimated by a disturbance observer. Next,
the estimated disturbance is compensated for in the control design.
The former approach frequently uses offline modeling based on historical
data, and the established model may not match the actual system. The
online estimation technique, however, can capture actual disturbances.
But a lot of factors can have an impact on the estimation inaccuracy.
As machine learning technology advances, online learning of complicated
disturbances offers a way to handle the disturbance problem. A significant
advancement in flight control has been achieved by the control based
on disturbance learning\cite{o2022neural}. Better anti-disturbance
performance can be anticipated if machine learning based online learning
of complicated disturbances could be implemented in the PDR. 

\textbf{(2) Image-based visual servo control }

Because of their high precision in close vicinity, machine vision
technologies are frequently used in the navigation system during the
refueling phase to determine the relative distance between the probe
and the drogue. If a vision-based navigation system is used, image-based
visual servo control is favored. Position-Based Visual Servo (PBVS)
and Image-Based Visual Servo (IBVS) are two categories for the current
visual servo control methods. The features in PBVS are a collection
of 3D parameters that must be estimated from image data. Once the
pose estimation is complete, the servo control can be executed. A
collection of 2D features that are instantly accessible in the image
data makes up the features in IBVS. AAR and robotic systems have benefited
from the widespread use of the visual servo control technique in recent
years. These works primarily pay attention to PBVS. Due to issues
with camera calibration, installation, and 3D object modeling errors,
a precise 2D picture observation does not necessarily indicate a precise
3D pose estimation. In order to obtain more accurate docking control,
it is therefore preferable to learn IBVS. Ref.99 examined the IBVS
control design for PDR. The use of an inner and outer loop controller
structure was made to achieve zero image error with the outer loop
visual servo controller and intended camera motion with the inner
loop stabilization controller. 

\subsubsection{Possible Future Work for high-efficiency requirement }

Compound control, simultaneous refueling of multiple UAVs, and global
refueling scheduling may be feasible future work for the high-efficiency
requirement. 

\textbf{(1) Compound control} 

High efficiency and precision are crucial for AAR. In general, feedforward
control can accomplish high efficiency, while feedback control can
ensure high precision. In order to achieve high precision while maintaining
excellent efficiency, compound control (feedforward control + feedback
control) is advised. \textquotedbl ILC (learning feedforward) + feedback
control\textquotedbl{} is one potential method. ILC is first designed
to mitigate the effects of repetitive uncertainties during the refueling
phase, and then feedback control can be further developed to mitigate
the effects of nonrepetitive uncertainties. For the \textquotedbl ILC
(learning feedforward) + feedback control\textquotedbl{} system, it
is necessary to study their respective control authorities. Furthermore,
the final contact velocity is crucial for AAR, but most study has
focused primarily on tracking trajectory rather than controlling velocity.
If both velocity control and trajectory control are taken into account,
the control issue will become an underactuated control problem, necessitating
the use of some unique control methods.

\textbf{(2) Multi-UAVs simultaneous refueling }

The demand for multiple aircraft refueling is quickly expanding due
to the development of swarm UAV systems. There are two possibilities
for refueling multiple aircraft. The first situation involves concurrently
refueling multiple receivers, while the second involves sequentially
refueling multiple receivers. With multiple refueling pods mounted
on a single tanker, PDR has the benefit of allowing up to three receivers
to be refueled concurrently. The majority of prior studies, however,
only considered one receiver instance and gave little thought to multiple
aircraft refueling. Multiple aircraft refueling requires fewer tankers
than the AAR of a single receiver. However, as a tanker and several
receivers are coupled aerodynamically, the technology involved becomes
much more complex. Future research should focus on the AAR for multi-UAVs
simultaneous refueling, which also involves cooperative control and
decision-making, management of flight paths, modeling for aerodynamic
impact from other receivers, and formation mechanism. 

\textbf{(3) Global refueling scheduling }

When refueling multiple receivers sequentially, global scheduling
is essential. AAR is required when several fighters work together
to complete a task over an extended period of time. Additional classifications
include \textquotedbl multiple receivers and a single tanker\textquotedbl{}
and \textquotedbl multiple receivers and multiple tankers\textquotedbl .
The problem of global scheduling gets more challenging when there
are more aircraft. The refueling sequence and path should be decided
to increase refueling efficiency under the precondition of assuring
mission requirements, which can be formulated as an optimization problem,
such as the least fuel consumption problem or shortest refueling time
problem. The best optimization techniques are then chosen to provide
a solution for the constrained optimization problem. Ref.\cite{hao2021autonomous}
looked at rendezvous scheduling for multiple aircraft' aerial refueling
technology. An integer linear programming method was developed to
accomplish the fastest refueling. 

\section{The Objective and Structure of the Book}

The book aims at solving modeling, control and decision-making problems
in the probe-and-drogue AAR. Chapter 1 introduces the significance
of AAR through discussing the following questions: why the probe-and-drogue
AAR is very important, what are the typical characteristics of such
a problem, what is the state-of-the-art technology, and what are the
difficulties faced. The remainder of this book has five parts,
sixteen chapters, as shown in Figure.\ref{Fig_1.12}. 

\begin{figure}
\begin{centering}
\includegraphics[width=1\textwidth]{Figures/Figs_Ch1/Fig_1\lyxdot 12}
\par\end{centering}
\caption{Structure of this book}

\centering{}\label{Fig_1.12}
\end{figure}

(i) Part I. Modeling Part 

Through this part, readers can have a deeper understanding of the
dynamic model of the probe-and-drogue AAR, which will be further used
for control in the following. This part contains four chapters, including
notation and systems of axes, airflow models, dynamics models of aircraft
and drogue and a simulation platform for probe-and-drogue AAR which
has connected all the models together for the following control and
decision-making design. These correspond to Chaps. 2\textendash 5,
respectively. 
\begin{itemize}
\item Chapter 2. Coordinate Systems 
\item Chapter 3 Aerodynamic Disturbances  
\item Chapter 4 Aircraft with Hose and Drogue
\item Chapter 5 AARSim: An Integrated High-fidelity Simulation Platform for Autonomous Aerial Refueling 
\end{itemize}
This part is mainly based on our work \cite{wei2016drogue},\cite{dai2016modeling},\cite{dai2019hose}.
The paper \cite{wei2016drogue} proposed a lower-order dynamic model
to describe drogue dynamics under the bow wave effect. The model consists
of two components: one is a second-order transfer function matrix
to describe the drogue dynamics, and the other is a nonlinear function
vector to describe the bow wave effect model. To make the modeling
easier, the paper \cite{dai2016modeling} analyzed the bow wave effect
and presented a simpler stream function based method to model it.
With the obtained aerodynamic coefficients, the induced aerodynamic
force on the drogue was calculated. We found that the drogue dynamics
model with the bow wave was not accurate enough, especially on the
vertical position. This is because the HDU is not involved. Since
the docking process is an accurate task, and a sub-meter level error
may results in a failed docking, in \cite{dai2019hose}, an improved integrated
model is proposed by considering the effect of HDU.

(ii) Part II. Navigation Part

Through this part, readers can have a deeper understanding of the relative position estimation in probe-and-drogue AAR by using vision technology. This corresponds to Chap. 6.
\begin{itemize}
	\item Chapter 6. Vision-Based Relative Position Estimation
\end{itemize}
This part is mainly based on our work \cite{2018Vision}.

{*}{*}Introduction to the paper above{*}{*}

(iii) Part III. Control Part 

Through this part, readers can have a deeper understanding of the
low-level control of the probe-and-drogue AAR, where the station-keeping
control and the docking control are proposed. Moreover, relative visual
navigation between the receiver and the drogue is proposed especially
for docking as it requires higher accuracy than the normal flight.
With them in hand, probe-and-drogue AAR can be guided by the decision-making
module which will be introduced in Part IV. These correspond to Chaps.
7\textendash 11 in the following. 
\begin{itemize}
\item Chapter 7 Additive-State-Decomposition-Based
Station-Keeping Control 
\item Chapter 8 Hose-Drum-Unit  Control 
\item Chapter 9 Terminal Iterative Learning Docking Control
\item Chapter 10 Improved Terminal Iterative Learning Docking Control  
\item Chapter 11 Image-Based Visual Servo Docking Control 
\end{itemize}
This part is mainly based on our work \cite{dai2018terminal},\cite{dai2018iterative},\cite{ren2019reliable}. 

{*}{*}Introduction to the papers above{*}{*}

(iv) Part IV. Safety Part 

Through this part, readers can have a deeper understanding about the
high-level decision-making of probe-and-drogue AAR, where a method
to determine the optimal trim state for aerial docking, a docking
success rate prediction method and a failsafe mechanism design based
on supervisory control theory are proposed. These correspond to Chaps.
12\textendash 14, respectively. 
\begin{itemize}
\item Chapter 12 Reachability Analysis on Optimal Trim State for Aerial Docking
\item Chapter 13 Docking Success Probability Prediction based on Stochastic Approximation
\item Chapter 14 Failsafe Mechanism Design using State Tree Structures
\end{itemize}
This part is mainly based on our work \cite{liu2021reachability},\cite{2021A},\cite{dong2019failsafe}. Aerial Refueling is an important method to increase
the endurance and flight range of aircraft, but it often suffers from
a low success rate. The altitude and speed of the tanker aircraft
in the docking phase have a great influence on the docking success
rate. According to this, the optimal trim state, namely the optimal
speed and altitude of the tanker aircraft, is investigated through
the reachability analysis method in \cite{liu2021reachability}. At the docking phase,
the docking risk is high as the receiver aircraft is approaching the
tanker aircraft. In order to guarantee the safety of the AAR process,
it is important to predict the docking success rate. Motivated by
this, a stochastic approximation method is adopted to evaluate the
docking success rate of the receiver aircraft by taking random disturbances
into account \cite{2021A}. Dangerous flight maneuvers may be executed when
unexpected failures or command conflicts happen. In order to solve
this problem, a decision-making logic with the failsafe mechanism
based on State Tree Structure (STS) is proposed to make the whole
flight safer in \cite{dong2019failsafe}.

(v) Part V. Planning Part 

Through this part, readers can have a deeper understanding of the scheduling and path planning of probe-and-drogue AAR. This corresponds to Chaps. 15-16.
\begin{itemize}
	\item Chapter 15 Autonomous Aerial Refueling of Multiple receivers:
	An Efficient Rendezvous Scheduling Approach
	\item Chapter 16 Aerial refueling scheduling of multi-receiver and multi-tanker under spatial-temporal constraints
\end{itemize} 
This part is mainly based on our work \cite{hao2021autonomous},\cite{huang2024aerial}.


