
\chapter*{Foreword}

Drones and new types of aerial vehicles are conquering the skies at
an impressive pace -- enabled by rapid progress in sensors, electric
components and embedded computing power. Whereas 20 years ago, development,
integration and testing of automated functions for aerospace application
were restricted to people working for large scale transport or military
aircraft, everybody can nowadays develop, build and test novel algorithms,
leveraging the power of freely available consumer Unmanned Aircraft
System (UAS) and open-source projects.

This development is nothing else than a revolution -- it empowers
any interested person to not only engage in the fascinating fields
of modeling, simulation, guidance, navigation and control of aerial
vehicles but also to turn own ideas into reality and actually fly
and test them.

So the big question is now -- how to get started?!? -- How to find
the right balance between theoretical depth and practical pragmatism?
-- How to be guided through the jungle of drivers, settings and wirings
of embedded hardware?

``Multicopter Design and Control Practice --- A Series Experiments
Based on MATLAB and Pixhawk``, the new book by Quan Quan, Xunhua
Dai and Shuai Wang of Beihang University, Beijing, China, is the appropriate
answer. After their successful book ``Introduction to Multicopter
Design and Control'' where the interesting reader can learn about
the theory of designing, simulating and controlling multicopter vehicles,
this new book builds on the powerful resources provided by the MathWorks
toolchains (MATLAB, Simulink, Stateflow, $\cdots$) as the backbone
for model-based system engineering and the Pixhawk with PX4 as the
most distributed open-source hardware and software solution for guidance,
navigation and control of low-cost aerial vehicles in a multitude
of configurations.

The book provides a detailed step-by-step tutorial describing the
Pixhawk system, setting up hardware, software, simulation and hardware
in the loop simulation. It details the experimental process and addresses
every involved component in an intuitive and tangible way. Sensor
calibration, state-estimation and filter design, attitude control,
position control and supervisory logic are presented in detailed experiments
step-by-step that all follow the scheme of introducing the experiment,
presenting preliminaries, theory, analysis and finally the design
task and a summary.

After successfully accomplishing the book by executing the practical
experiments by oneself, the reader is capable of developing and more
important deploying and testing algorithms for multicopters and other
types of drones.

The fascinating world of guiding, navigation and controlling aerial
vehicles is no longer a privilege of a small group -- it is accessible
to everyone. Even if you never interacted with embedded hardware and
software before, the book empowers you, to become a part of it. It
gives you the hard skills required to make things happen.

\qquad{}\qquad{}\qquad{}\qquad{}\qquad{}\qquad{}\qquad{}\qquad{}\qquad{}\qquad{}\qquad{}\qquad{}\qquad{}Florian
Holzapfel 

\qquad{}\qquad{}\qquad{}\qquad{}\qquad{}\qquad{}\qquad{}\qquad{}\qquad{}\qquad{}\qquad{}Institute
of Flight System Dynamics 

\qquad{}\qquad{}\qquad{}\qquad{}\qquad{}\qquad{}\qquad{}\qquad{}\qquad{}\qquad{}Technical
University of Munich, Germany 

\qquad{}\qquad{}\qquad{}\qquad{}\qquad{}\qquad{}\qquad{}\qquad{}\qquad{}\qquad{}\qquad{}\qquad{}\qquad{}November
2019
