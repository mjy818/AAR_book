
\chapter{Coordinate Systems}
\label{chap2}
In order to describe the attitude and position of aircraft and refueling
equipment in aerial refueling systems, it is necessary to establish
appropriate coordinate frames. Especially in the modeling process,
several frames need to be introduced to describe the relative and
absolute states of objects in the system, such as the absolute position
and attitude of aircraft, the relative motion among aircraft and refueling
equipment, and the forces and moments acting on the aircraft. These
frames will also be beneficial to the simplification of analysis and
controller design for aerial refueling systems. Besides, choosing
an appropriate coordinate system can effectively simplify the establishment
and solution of the dynamics and kinematics equations of the aircraft,
as well as conveniently describe the relative motion between the probe
and the drogue during the aerial refueling process, especially in
the docking phase. This chapter mainly introduces several frames commonly
used in aerial refueling, along with the mutual conversion relationship
among them.

\section{Definitions and Notation}

\subsection{Right-Hand Rule}

In the beginning, the \textbf{right-hand rule} needs to be introduced
for the definitions of different coordinate frames. A typical coordinate
definition in line with the right-hand rule is presented in Fig. \ref{Fig_2.1}(a),
where the right thumb points to the positive direction of the \textit{ox}
axis, the first finger points to the positive direction of the \textit{oy}
axis, and the middle finger points to the positive direction of the
\textit{oz} axis. Furthermore, the right-hand rule is also widely
used in determining the positive direction of a rotation transformation
as shown in Fig. \ref{Fig_2.1}(b), where the thumb of the right-hand
points the positive direction of the rotation axis and the direction
of the bent fingers is the positive direction of the rotation. All
the frames and rotation transformations used in this book are defined
by the right-hand rule. 

\begin{figure}
	\begin{centering}
		\includegraphics[width=0.8\textwidth]{Figures/Figs_Ch2/Fig_2\lyxdot 1}
		\par\end{centering}
	\caption{Definitions of coordinate axes and positive rotation transformation
		with the right-hand rule}
	
	\centering{}\label{Fig_2.1}
\end{figure}


\subsection{Variable Notation}

In this paper, scalar variables are represented by normal italic letters,
such as the air density $\rho\in\mathbb{R}$; vector variables are
represented by bold lowercase letters, such as the velocity vector
$\mathbf{v}\in{{\mathbb{R}}^{3}}$ ; matrix variables are represented
by bold capital letters, such as the rotation matrix $\mathbf{R}\in{{\mathbb{R}}^{3\times3}}$.
The superscript or subscript of a variable indicates the concrete
meaning of the variable. For example, ${{\mathbf{p}}_{\text{d}}}$
denotes the position vector of the drogue, where \textquotedblleft d\textquotedblright{}
is the abbreviation of the word \textquotedblleft drogue\textquotedblright .
The abbreviation symbols commonly-used in this book are shown in Table
\ref{Tab_2.1}.

\begin{table}
	\caption{Implication of abbreviation symbols of superscripts and subscripts}
	
	\begin{centering}
		\begin{tabular}{|c|c|}
			\hline 
			Abbreviation symbols & Implication\tabularnewline
			\hline 
			d & drogue\tabularnewline
			\hline 
			e & equilibrium position of drogue\tabularnewline
			\hline 
			f & origin of CFD frame\tabularnewline
			\hline 
			g & ground\tabularnewline
			\hline 
			h & hose\tabularnewline
			\hline 
			n & nose of aircraft head\tabularnewline
			\hline 
			p & probe\tabularnewline
			\hline 
			r & receiver\tabularnewline
			\hline 
			t & tanker\tabularnewline
			\hline 
			w & wind\tabularnewline
			\hline 
		\end{tabular}
		\par\end{centering}
	\centering{}\label{Tab_2.1}
\end{table}

Based on the literature \cite{AirContrl}, the following notation
is used to describe the motion of objects in aerial refueling systems:

(1) A right superscript on a vector specifies a frame. It denotes
a vector and its components are defined in the specified frame. For
example, ${{\mathbf{p}}^{\text{g}}}=[\begin{array}{ccc}
{{x}^{\text{g}}} & {{y}^{\text{g}}} & {{z}^{\text{g}}}\end{array}]{}^{\text{T}}$ denotes the position vector in the ground frame (abbreviated as ``g''
).

(2) Right subscripts are used to designate the meaning of a variable.
For example, ${{\mathbf{p}}_{\text{t}}}$ denotes the position vector
(expressed by bold lowercase letter ``$\mathbf{p}$'') of the tanker
aircraft (abbreviated as ``t''). Besides, a \textquotedblleft /\textquotedblright{}
in a subscript means \textquotedblleft with respect to\textquotedblright ,
which represents the relative relationship between two objects. For
example, ${{\mathbf{p}}_{\text{p/r}}}$ denotes the position vector
of the probe position (abbreviated as ``p'') with respect to the
mass center point of the receiver aircraft (abbreviated as ``r'').

(3) When a variable is an angle (or rotation matrix), the right subscript
\textquotedblleft A/B\textquotedblright{} indicates the angular relationship
of the frame A relative to frame B. For example, ${{\mathbf{R}}_{\text{r/t}}}$
denotes the rotation matrix to describe the angular relationship of
the receiver frame ``r'' relative to the tanker frame ``t'', which
is also used to describe the transformation of a vector from the tanker
frame ``t'' to the receiver frame ``r''. 

(4) The components of a vector on the three axes are expressed by
a $3\times1$ array with the subscript inherited from that vector.
For example, ${{\mathbf{v}}_{\text{t}}}={{[\begin{array}{ccc}
		{{v}_{\text{t},x}} & {{v}_{\text{t},y}} & {{v}_{\text{t},z}}\end{array}]}^{\text{T}}}$ represents the three components of the tanker velocity. In order
to simplify the derivation process, an exception is made for the most
frequently-used variables -- the position vectors, whose components
are simply expressed as ${{\mathbf{p}}_{\text{t}}}={{[\begin{array}{ccc}
		{{x}_{\text{t}}} & {{y}_{\text{t}}} & {{z}_{\text{t}}}\end{array}]}^{\text{T}}}$.

(5) When the reference frame is the tanker frame (see the detailed
definition in the next section), the superscript ``t'' and subscript
``/t'' of a variable can be omitted for simplicity.
This is because, during the docking modeling process, the reference
frame for most variables is the tanker frame. For example, $\mathbf{p}_{\text{d/r}}^{\text{t}}={{[\begin{array}{ccc}
		x_{\text{d/r}}^{\text{t}} & y_{\text{d/r}}^{\text{t}} & z_{\text{d/r}}^{\text{t}}\end{array}]}^{\text{T}}}$ represents the relative position between the drogue (``d'') and
the receiver aircraft ``r'', and the relative position vector is
projected in the tanker frame (``t''), then it can be abbreviated
as ${{\mathbf{p}}_{\text{d/r}}}={{[\begin{array}{ccc}
		{{x}_{\text{d/r}}} & {{y}_{\text{d/r}}} & {{z}_{\text{d/r}}}\end{array}]}^{\text{T}}}$ for simplicity. In addition, ${{\Theta}_{\text{r}/\text{t}}}$ represents
the angular relationship between the receiver frame ``r'' and the
tanker frame ``t'', then it can be abbreviated as ${{\Theta}_{\text{r}}}$.

\section{Coordinate Frames}

There are five coordinate frames commonly used in kinematics and dynamics
modeling of aircraft: the ground frame, the body frame, the wind frame
and the stability frame. Meanwhile, the CFD frame and the drogue equilibrium
position frame (referred to as the drogue frame) will also be introduced
for presenting the relative relationship among the drogue, the probe
and the receiver aircraft. The frames used in the docking process
are shown in Fig. \ref{Fig_2.2}. These frames are detailed below
and the function of each coordinate system and the relationship among
them are explained.

\begin{figure}
	\begin{centering}
		\includegraphics[width=0.8\textwidth]{Figures/Figs_Ch2/Fig_2\lyxdot 2}
		\par\end{centering}
	\caption{Coordinate frames used in the docking process}
	
	\centering{}\label{Fig_2.2}
\end{figure}


\subsection{The Ground Frame}

The ground frame ${{o}_{\text{g}}}-{{x}_{\text{g}}}{{y}_{\text{g}}}{{z}_{\text{g}}}$
(also called the flat-earth frame) is defined by ignoring the earth\textquoteright s
curvature on the ground (the earth\textquoteright s surface is assumed
to be flat in a small region), so the ground frame can be treated
as an inertial system. The above assumption is also referred to as
the flat earth assumption in many references. This assumption is reasonable
for this book because the aerial refueling process is operated in
a limited region. As shown in Fig. \ref{Fig_2.3}, the coordinate
origin ${{o}_{\text{g}}}$ of the ground frame is one fixed point
on the ground which is usually selected as the take-off position of
an aircraft; the axis ${{o}_{\text{g}}}{{x}_{\text{g}}}$ usually
points to the north or points to the horizontal direction of the aircraft
motion for convenience; the axis ${{o}_{\text{g}}}{{z}_{\text{g}}}$
points vertically downwards to the ground; the axis ${{o}_{\text{g}}}{{y}_{\text{g}}}$
is determined according to the right-hand rule.

\begin{figure}
	\begin{centering}
		\includegraphics[width=0.5\textwidth]{Figures/Figs_Ch2/Fig_2\lyxdot 3}
		\par\end{centering}
	\caption{Definition of the ground frame}
	
	\centering{}\label{Fig_2.3}
\end{figure}


\subsection{The Body Frames}

A coordinate frame fixed to the aircraft body can be called the body
frame. The traditional body frame ${{o}_{\text{b}}}-{{x}_{\text{b}}}{{y}_{\text{b}}}{{z}_{\text{b}}}$
usually selects the mass center of an aircraft as the origin ${{o}_{\text{b}}}$,
and its axis directions are defined as: the axis ${{o}_{\text{b}}}{{x}_{\text{b}}}$
points to the nose direction in the symmetric plane of the aircraft;
the axis ${{o}_{\text{b}}}{{y}_{\text{b}}}$ is perpendicular to the
symmetric plane and points to the right side; the axis ${{o}_{\text{g}}}{{z}_{\text{g}}}$
is vertical to the axis ${{o}_{\text{b}}}{{x}_{\text{b}}}$ in the
symmetric plane and points to below, which satisfies the right-hand
rule. The moment of inertia of an aircraft is usually defined in the
above body frame to describe its rotational motion produced by the
forces and moments (also described in the body frame) acting on the
aircraft. Since there are a tanker aircraft and several receiver aircraft
in an aerial refueling system, their body frames are distinguished
by the tanker body frame and the receiver body frame, respectively.

\subsubsection{The Tanker Body Frame}

The tanker body frame ${{o}_{\text{t}}}-{{x}_{\text{t}}}{{y}_{\text{t}}}{{z}_{\text{t}}}$
is also simplified as the taker frame for short. During the aerial
refueling process, the tanker aircraft is commanded to fly straight
and level with a constant speed in most situations, so the tanker
frame can be assumed to be a moving inertial coordinate frame. For
the convenience of describing the movement of the hose, the drogue,
and the receivers at the same time, the origin of the tanker body
frame is selected as the connection point between the hose and the
tanker fuselage as shown in Fig. \ref{Fig_2.4}, which is different
from the definition of traditional body frame whose origin is on the
mass center. Then, the axis directions of the tanker frame are defined
as follows: the axis ${{o}_{\text{t}}}{{x}_{\text{t}}}$ points to
the horizontal direction of the tanker ground velocity $\mathbf{v}_{\text{t}}^{\text{g}}$;
the axis ${{o}_{\text{t}}}{{z}_{\text{t}}}$ points vertically downwards
to the ground; the ${{o}_{\text{t}}}{{y}_{\text{t}}}$ axis is determined
according to the right-hand rule.

The relative position and attitude between the drogue and the receiver
are mainly concerned in the docking phase, where the tanker frame
(moving inertial frame) is more convenient for modeling and analysis
than the ground frame (fixed inertial frame). In order to further
simplify the coordinate transformation among different frames, the
direction of the axis ${{o}_{\text{g}}}{{x}_{\text{g}}}$ of the ground
frame is also selected the same as the horizontal moving direction
of the tanker, so the tanker frame and the ground frame have the same
direction definitions and the rotation transformation between them
can be avoided. 

Since both the ground frame and the tanker frame are inertial frames,
we can use the tanker frame to describe all the relative motion among
objects in the aerial refueling system, and use the ground frame to
describe their absolute motion relative to the ground. 

\begin{figure}
	\begin{centering}
		\includegraphics[width=0.6\textwidth]{Figures/Figs_Ch2/Fig_2\lyxdot 4}
		\par\end{centering}
	\caption{The tanker body frame}
	
	\centering{}\label{Fig_2.4}
\end{figure}


\subsubsection{The Receiver Body Frame}

The receiver body frame ${{o}_{\text{r}}}-{{x}_{\text{r}}}{{y}_{\text{r}}}{{z}_{\text{r}}}$
(or simply \textquotedblleft the receiver frame\textquotedblright )
is used to describe the physical quantities associated with the attitude
angle of the receiver. As shown in Fig. \ref{Fig_2.5}, the origin
${{o}_{\text{r}}}$ of the receiver frame is fixed to the mass center
of the receiver ${{\mathbf{p}}_{\text{r}}}$, and the direction of
the coordinate axis is consistent with the traditional body frame,
where the ${{o}_{\text{r}}}{{x}_{\text{r}}}$ axis points to the nose
direction in the symmetric plane of the aircraft; the ${{o}_{\text{r}}}{{y}_{\text{r}}}$
axis is perpendicular to the symmetric plane and points to the right
side; the ${{o}_{\text{r}}}{{z}_{\text{r}}}$ axis is vertical to
the ${{o}_{\text{r}}}{{x}_{\text{r}}}$ axis in the symmetric plane
and points to below, which satisfies the right-hand rule.

\begin{figure}
	\begin{centering}
		\includegraphics[width=0.45\textwidth]{Figures/Figs_Ch2/Fig_2\lyxdot 5}
		\par\end{centering}
	\caption{The relationship between the receiver body frame and the stability frame}
	
	\centering{}\label{Fig_2.5}
\end{figure}


\subsection{The Stability Frame}

The stability frame ${{o}_{\text{s}}}-{{x}_{\text{s}}}{{y}_{\text{s}}}{{z}_{\text{s}}}$
is also widely used in the study of the motion characteristics of
the aircraft after a small disturbance during steady flight. The stability
frame is defined according to the airspeed vector of an aircraft which
is defined as the relative velocity of the aircraft to the surrounding
air. The tanker's airspeed vector is represented by ${{\mathbf{v}}_{\text{t}}}$,
and the receiver's airspeed vector is expressed as ${{\mathbf{v}}_{\text{r}}}$.

As shown in Fig. \ref{Fig_2.5}, taking the receiver as an example,
the origin ${{o}_{\text{s}}}$ of the receiver stability frame is
fixed to the mass center of the receiver; the axis ${{o}_{\text{s}}}{{x}_{\text{s}}}$
coincides with the projection within the symmetric plane of the trimmed
airspeed vector $\mathbf{v}_{\text{r}}^{*}$ (the trim methods will
be introduced in Chapter 4) based on ${{\mathbf{v}}_{\text{r}}}$;
the axis ${{o}_{\text{s}}}{{z}_{\text{s}}}$ is perpendicular to the
axis ${{o}_{\text{s}}}{{x}_{\text{s}}}$ in the symmetric plane of
the aircraft and points vertically downwards to the ground; ${{o}_{\text{s}}}{{y}_{\text{s}}}$
points to the right and satisfies the right-hand rule.

\subsection{The Wind Frame}

The wind frame ${{o}_{\text{w}}}-{{x}_{\text{w}}}{{y}_{\text{w}}}{{z}_{\text{w}}}$
is very important in aerodynamic modeling of aircraft. The origin
${{o}_{\text{w}}}$ of the wind frame is fixed to the mass center
of the aircraft; the axis ${{o}_{\text{w}}}{{x}_{\text{w}}}$ coincides
with the trimmed airspeed vector $\mathbf{v}_{\text{a}}^{*}$; the
axis ${{o}_{\text{w}}}{{z}_{\text{w}}}$ is perpendicular to the axis
${{o}_{\text{w}}}{{x}_{\text{w}}}$ in the symmetric plane of the
aircraft and points vertically downwards to the ground; the axis ${{o}_{\text{w}}}{{y}_{\text{w}}}$
points to the right side and satisfies the right-hand rule.

The aerodynamic angle is determined by the relationship between the
wind frame and the body frame. As shown in Fig. \ref{Fig_2.6}, the
aerodynamic angles are defined as follows.

(1) \textbf{Angle of attack} $\alpha$: the angle between the projection
of the axis ${{o}_{\text{w}}}{{x}_{\text{w}}}$ of the wind frame
and the axis ${{o}_{\text{b}}}{{x}_{\text{b}}}$ of the body frame
on the symmetric plane of the aircraft, which is equal to the angle
between ${{o}_{\text{b}}}{{x}_{\text{b}}}$ and ${{o}_{\text{s}}}{{x}_{\text{s}}}$.
Note that the angle is positive when the projection of ${{o}_{\text{w}}}{{x}_{\text{w}}}$
is below ${{o}_{\text{b}}}{{x}_{\text{b}}}$.

(2) \textbf{Sideslip Angle} $\beta$: the angle between the ${{o}_{\text{w}}}{{x}_{\text{w}}}$
axis of the wind frame and the symmetric plane of the aircraft, which
is equal to the angle between ${{o}_{\text{w}}}{{x}_{\text{w}}}$
and ${{o}_{\text{s}}}{{x}_{\text{s}}}$. Note that the angle is positive
when the projection of ${{o}_{\text{w}}}{{x}_{\text{w}}}$ is on the
right of the symmetric plane of the aircraft.

\begin{figure}
	\begin{centering}
		\includegraphics[width=0.5\textwidth]{Figures/Figs_Ch2/Fig_2\lyxdot 6}
		\par\end{centering}
	\caption{The aerodynamic angles}
	
	\centering{}\label{Fig_2.6}
\end{figure}

The flight path angle and the course angle are determined by the relationship
between the wind frame and the ground frame. As shown in Fig. \ref{Fig_2.7},
taking the receiver as an example, the flight path angle and the course angle are defined
as follows.

(1) \textbf{Flight Path Angle} $\gamma$: The angle between the flight
ground velocity ${{\mathbf{v}}_{\text{b/g}}}$ (the velocity of the
aircraft relative to the ground) and the horizontal plane, which is
positive when the aircraft is flying upwards.

(2) \textbf{Course Angle} $\chi$: The angle between the projection
of the flight velocity vector ${{\mathbf{v}}_{\text{{b}/{g}}}}$ on
the horizontal plane and the axis ${{o}_{\text{g}}}{{x}_{\text{g}}}$
of the ground frame, which is positive when the projection is on the
right side of ${{o}_{\text{g}}}{{x}_{\text{g}}}$.

\begin{figure}
	\begin{centering}
		\includegraphics[width=0.5\textwidth]{Figures/Figs_Ch2/Fig_2\lyxdot 7}
		\par\end{centering}
	\caption{The flight path angle and the course angle}
	
	\centering{}\label{Fig_2.7}
\end{figure}


\subsection{The CFD Frame}

The CFD frame ${{o}_{\text{f}}}-{{x}_{\text{f}}}{{y}_{\text{f}}}{{z}_{\text{f}}}$
is defined for the convenience of studying the aerodynamic interference
(also called the bow wave effect) between the receiver forebody and
the drogue. As shown in Fig. \ref{Fig_2.8}, the direction of the
CFD frame is the same as the receiver frame while the origin is different.
The origin of the CFD frame is a point ${{\mathbf{p}}_{\text{f}}}$
in the plane ${{x}_{\text{r}}}{{o}_{\text{r}}}{{z}_{\text{r}}}$,
and its height is the same as the mounting position of the probe.
The reasons for taking ${{\mathbf{p}}_{\text{f}}}$ as the coordinate
origin are as follows. 

(1) The height of the axis ${{o}_{\text{f}}}{{z}_{\text{f}}}$ is
consistent with the height of the drogue, because the probe is aligned
with the drogue during the docking process, that is, the docking process
is completed near this height. 

(2) The coordinate origin is selected at the rear of the probe because
the pilot's view is roughly at this position, and this position is
also convenient for CFD computation.

\begin{figure}
	\begin{centering}
		\includegraphics[width=0.8\textwidth]{Figures/Figs_Ch2/Fig_2\lyxdot 8}
		\par\end{centering}
	\caption{Three views of the CFD frame}
	
	\centering{}\label{Fig_2.8}
\end{figure}


\subsection{The Drogue Equilibrium Position Frame}

The drogue equilibrium position frame ${{o}_{\text{e}}}-{{x}_{\text{e}}}{{y}_{\text{e}}}{{z}_{\text{e}}}$
(or simply \textquotedblleft the drogue frame\textquotedblright )
is defined for better evaluating the disturbed motion and docking
error of the probe and the drogue during the docking phase. Since
the tanker is assumed to fly at a uniform speed in a straight line,
the hose-drogue system will finally reach an equilibrium state relative
to the tanker body. Then, the drogue position (the center of the drogue
canopy) will also reach an equilibrium position or slightly fluctuate
around an equilibrium position under disturbances. This equilibrium
position (marked with ${{\mathbf{p}}_{\text{e}}}$) is selected as
the origin of the drogue frame, which is fixed to the tanker frame
and will not move as the drogue moves. As shown in Fig. \ref{Fig_2.9},
the axis directions of the drogue frame are consistent with the tanker
frame. Noteworthy, under different flight altitudes and speeds, the
drogue equilibrium position ${{\mathbf{p}}_{\text{e}}}$ may be different,
but it can be estimated before the docking phase starts.

\begin{figure}
	\begin{centering}
		\includegraphics[width=0.5\textwidth]{Figures/Figs_Ch2/Fig_2\lyxdot 9}
		\par\end{centering}
	\caption{The drogue equilibrium position frame}
	
	\centering{}\label{Fig_2.9}
\end{figure}


\section{Frame Transformation}

There are several ways to realize the frame transformation, or to
project a vector from one frame to another. Common methods used in
aerial refueling systems are: Euler angle method and Direction Cosine
Matrix (DCM) method. The two methods will be described separately
below.

\subsection{Euler Angle Method}

\subsubsection{Euler Angle Definitions}

Based on Euler\textquoteright s theorem, the rotation of a rigid body
around one fixed point can be regarded as the composition of several
finite rotations around that fixed point. The orientation of a body
frame can be achieved by three elemental rotations from the ground
frame around one fixed point (usually the mass center). During these
elemental rotations, each rotation axis is one of the coordinate axes
of the rotated frame, and each rotation angle is defined by one Euler
angle. It should be noted that the order of rotation is important,
and different rotation orders will result in different frames. For
an aircraft, the sequence of rotation axes \textit{z-y-x} is often
used to define its Euler angles.

The attitude of an aircraft, which is often described by the Euler
angles, is determined by the relationship between the body frame and
the ground frame. As shown in Fig. \ref{Fig_2.10}, ${{o}_{\text{g}}}-{{x}_{\text{g}}}{{y}_{\text{g}}}{{z}_{\text{g}}}$
represents the ground frame and ${{o}_{\text{b}}}-{{x}_{\text{b}}}{{y}_{\text{b}}}{{z}_{\text{b}}}$
represents the body frame. Euler angles are defined as follows.

(1) \textbf{Yaw angle} $\psi$: it is the angle between the axis ${{o}_{\text{g}}}{{x}_{\text{g}}}$
and the projection line from the axis ${{o}_{\text{b}}}{{x}_{\text{b}}}$
to the plane ${{o}_{\text{g}}}{{x}_{\text{g}}}{{y}_{\text{g}}}$.
The yaw angle is positive when the aircraft turns to the right, and
its value range is $\left[-\pi,\pi\right]$. 

(2) \textbf{Pitch angle} $\theta$: it is the angle between the axis ${{o}_{\text{b}}}{{x}_{\text{b}}}$
and the plane ${{o}_{\text{g}}}{{x}_{\text{g}}}{{y}_{\text{g}}}$. The
pitch angle is positive when the aircraft nose pitches up, and its
value range is $\left[-{\pi}/2,{{\pi}/{2}}\right]$. 

(3) \textbf{Roll angle} $\phi$: it is the angle between the axis ${{o}_{\text{b}}}{{z}_{\text{b}}}$
and the plane ${{o}_{\text{b}}}{{x}_{\text{b}}}{{{x}^{\prime}}_{\text{b}}}$.
The roll angle is positive when the aircraft rolls to the right, and
its value range is $\left[-{\pi}/2,{{\pi}/{2}}\right]$. 

\subsubsection{Vector Conversion}

\begin{figure}
	\begin{centering}
		\includegraphics[width=0.5\textwidth]{Figures/Figs_Ch2/Fig_2\lyxdot 10}
		\par\end{centering}
	\caption{The relationship between the body frame and the ground frame}
	
	\centering{}\label{Fig_2.10}
\end{figure}

This section introduces the conversion relationship of vectors among
different frames. First of all, vector conversions from the ground
frame to the body frame will be presented. As shown in Fig. \ref{Fig_2.11},
the rotation is composed of three elemental rotations around axes
$o{{z}_\text{g}}$, $o{{k}_{2}}$, and $o{{n}_{1}}$ by angles $\psi,\theta,\phi$,
respectively. More specifically, we first move the origin ${{o}_{\text{b}}}$
of the body frame and the origin ${{o}_{\text{g}}}$ of the ground
frame to the same point $o$, so the body frame and ground frame are
presented by $o-{{x}_{\text{b}}}{{y}_{\text{b}}}{{z}_{\text{b}}}$
and $o-{{x}_{\text{g}}}{{y}_{\text{g}}}{{z}_{\text{g}}}$ respectively,
and then perform the following operations.

(1) As shown in Fig. \ref{Fig_2.11}(a), rotate the ground frame $o-{{x}_{\text{g}}}{{y}_{\text{g}}}{{z}_{\text{g}}}$
around the axis $o{{z}_{\text{g}}}$ by the yaw angle $\psi$ to obtain
a new coordinate frame $o-{{k}_{1}}{{k}_{2}}{{k}_{3}}$, so that $o{{x}_{\text{g}}}$
turns to $o{{k}_{1}}$ and $o{{y}_{\text{g}}}$ turns to $o{{k}_{2}}$.
Thus, the axis $o{{k}_{1}}$ is located in the plane $o{{x}_{\text{b}}}{{z}_{\text{b}}}$
of the body frame, $o{{k}_{2}}$ is located in the plane $o{{y}_{\text{b}}}{{z}_{\text{b}}}$. 

(2) As shown in Fig. \ref{Fig_2.11}(b), rotate the frame $o-{{k}_{1}}{{k}_{2}}{{k}_{3}}$
around the axis $o{{k}_{2}}$ by the pitch angle $\theta$ to obtain
a new coordinate frame $o-{{n}_{1}}{{n}_{2}}{{n}_{3}}$. Thus, $o{{n}_{1}}$
corresponds to the axis $o{{x}_{\text{b}}}$ of the body frame, $o{{n}_{3}}$
is located in the plane $o{{y}_{\text{b}}}{{z}_{\text{b}}}$.

(3) As shown in Fig. \ref{Fig_2.11}(c), rotate the frame $o-{{n}_{1}}{{n}_{2}}{{n}_{3}}$
around the axis $o{{n}_{1}}$ by the roll angle $\phi$ to obtain
body frame $o-{{x}_{\text{b}}}{{y}_{\text{b}}}{{z}_{\text{b}}}$,
so that $o{{n}_{2}}$ turns to the axis $o{{y}_{\text{b}}}$ of the
body frame, $o{{n}_{3}}$ turns to $o{{z}_{\text{b}}}$ of the body
frame. 

\begin{figure}
	\begin{centering}
		\includegraphics[width=0.8\textwidth]{Figures/Figs_Ch2/Fig_2\lyxdot 11}
		\par\end{centering}
	\caption{Euler angles and frame transformation}
	
	\centering{}\label{Fig_2.11}
\end{figure}

The detailed mathematical derivations of the above procedures are
presented as follows.

(1) After rotating the yaw angle $\psi$ around $o{{z}_{\text{g}}}$
as shown in Fig. \ref{Fig_2.11}(a), the following equations can be
obtained 
\begin{equation}
\left\{ \begin{array}{l}
{\mathbf{k}_{1}}={{\mathbf{i}}^{\text{g}}}\cos\psi+{{\mathbf{j}}^{\text{g}}}\sin\psi\\
{{\mathbf{k}}_{2}}=-{{\mathbf{i}}^{\text{g}}}\sin\psi+{{\mathbf{j}}^{\text{g}}}\cos\psi\\
{{\mathbf{k}}_{3}}={{\mathbf{k}}^{\text{\ensuremath{\text{g}}}}}
\end{array}\right.\label{eq:2.1}
\end{equation}
where ${{\mathbf{i}}^{\text{g}}}$ is the unit vector along $o{{x}_{\text{g}}}$,
${{\mathbf{j}}^{\text{g}}}$ is the unit vector along $o{{y}_{\text{g}}}$,
and ${{\mathbf{k}}^{\text{g}}}$ is the unit vector along $o{{z}_{\text{g}}}$.
Note that, the bold symbols $\mathbf{k}_{1},\mathbf{k}_{2},\mathbf{k}_{3}$
represent the axial unit vectors of the new frame $o-{{k}_{1}}{{k}_{2}}{{k}_{3}}$.
Thus, the transformation relationship between $o-{{k}_{1}}{{k}_{2}}{{k}_{3}}$
and $o-{{x}_{\text{g}}}{{y}_{\text{g}}}{{z}_{\text{g}}}$ is 
\begin{equation}
\left[\begin{array}{l}
{k_{1}}\\
{k_{2}}\\
{k_{3}}
\end{array}\right]=\left[{\begin{array}{ccc}
	{\cos\psi} & {\sin\psi} & 0\\
	{-\sin\psi} & {\cos\psi} & 0\\
	0 & 0 & 1
	\end{array}}\right]\left[{\begin{array}{c}
	{x_{\text{g}}}\\
	{y_{\text{g}}}\\
	{z_{\text{g}}}
	\end{array}}\right].\label{eq:2.2}
\end{equation}
where $[\begin{array}{ccc}
x_{\text{g}} & y_{\text{g}} & z_{\text{g}}\end{array}]^{\text{T}}$ denotes the coordinate values of a vector projected (presented, described
or observed) in the frame $o-{{x}_{\text{g}}}{{y}_{\text{g}}}{{z}_{\text{g}}}$,
and $[\begin{array}{ccc}
k_{1} & k_{2} & k_{3}\end{array}]^{\text{T}}$ denotes the new coordinate values of that vector projected in the
frame $o-{{k}_{1}}{{k}_{2}}{{k}_{3}}$.

By letting
\begin{equation}
{{\mathbf{R}}_{\text{z}}}\left(\psi\right)\triangleq\left[\begin{array}{ccc}
\cos\psi & \sin\psi & 0\\
-\sin\psi & \cos\psi & 0\\
0 & 0 & 1
\end{array}\right]\label{eq:2.3}
\end{equation}
then Eq. (\ref{eq:2.2}) is rewritten as 
\begin{equation}
\left[\begin{array}{l}
{k_{1}}\\
{k_{2}}\\
{k_{3}}
\end{array}\right]={\mathbf{R}_{z}}\left(\psi\right)\left[{\begin{array}{c}
	{x_{\text{g}}}\\
	{y_{\text{g}}}\\
	{z_{\text{g}}}
	\end{array}}\right].\label{eq:2.4}
\end{equation}

(2) After rotating the pitch angle $\theta$ around $o{{k}_{2}}$
as shown in Fig. \ref{Fig_2.11}(b), the following equations can be
obtained
\begin{equation}
\left\{ \begin{array}{l}
{{\mathbf{n}}_{1}}={{\mathbf{k}}_{1}}\cos\theta-{{\mathbf{k}}_{3}}\sin\theta\\
{{\mathbf{n}}_{2}}={{\mathbf{k}}_{2}}\\
{{\mathbf{n}}_{3}}={{\mathbf{k}}_{1}}\sin\theta+{{\mathbf{k}}_{3}}\cos\theta
\end{array}\right.\label{eq:2.5}
\end{equation}
where the bold symbols $\mathbf{n}_{1},\mathbf{n}_{2},\mathbf{n}_{3}$
represent the axial unit vectors of the new frame $o-{{n}_{1}}{{n}_{2}}{{n}_{3}}$.
Thus, the transformation relationship between $o-{{n}_{1}}{{n}_{2}}{{n}_{3}}$
and $o-{{k}_{1}}{{k}_{2}}{{k}_{3}}$ is given by
\begin{equation}
\left[{\begin{array}{c}
	{n_{1}}\\
	{n_{2}}\\
	{n_{3}}
	\end{array}}\right]=\left[{\begin{array}{ccc}
	{\cos\theta} & 0 & {-\sin\theta}\\
	0 & 1 & 0\\
	{\sin\theta} & 0 & {\cos\theta}
	\end{array}}\right]\left[\begin{array}{l}
{k_{1}}\\
{k_{2}}\\
{k_{3}}
\end{array}\right].\label{eq:2.6}
\end{equation}
By letting 
\begin{equation}
{{\mathbf{R}}_{y}}\left(\theta\right)\triangleq\left[\begin{array}{ccc}
\cos\theta & 0 & -\sin\theta\\
0 & 1 & 0\\
\sin\theta & 0 & \cos\theta
\end{array}\right]\label{eq:2.7}
\end{equation}
one has
\begin{equation}
\left[{\begin{array}{c}
	{n_{1}}\\
	{n_{2}}\\
	{n_{3}}
	\end{array}}\right]={\mathbf{R}_{y}}\left(\theta\right)\left[\begin{array}{l}
{k_{1}}\\
{k_{2}}\\
{k_{3}}
\end{array}\right]={\mathbf{R}_{y}}\left(\theta\right){\mathbf{R}_{z}}\left(\psi\right)\left[{\begin{array}{c}
	{x_{\text{g}}}\\
	{y_{\text{g}}}\\
	{z_{\text{g}}}
	\end{array}}\right]\label{2.8}
\end{equation}

(3) After rotating the roll angle $\phi$ around $o{{n}_{1}}$ as
shown in Fig. \ref{Fig_2.11}(c), the following equation can be obtained
\begin{equation}
\left\{ \begin{array}{l}
{{\mathbf{x}}_{\text{b}}}={{\mathbf{n}}_{1}}\\
{{\mathbf{y}}_{\text{b}}}={{\mathbf{n}}_{2}}\cos\phi+{{\mathbf{n}}_{3}}\sin\phi\\
{{\mathbf{z}}_{\text{b}}}=-{{\mathbf{n}}_{2}}\sin\phi+{{\mathbf{n}}_{3}}\cos\phi
\end{array}\right.\label{eq:2.9}
\end{equation}
where ${\mathbf{x}}_{\text{b}},{\mathbf{y}}_{\text{b}},{\mathbf{z}}_{\text{b}}$
represent the axial unit vectors of the finally obtained aircraft
body frame $o-{{x}_{\text{b}}}{{y}_{\text{b}}}{{z}_{\text{b}}}$.
Thus, the transformation relationship between $o-{{x}_{\text{b}}}{{y}_{\text{b}}}{{z}_{\text{b}}}$
and $o-{{n}_{1}}{{n}_{2}}{{n}_{3}}$ is
\begin{equation}
\left[\begin{array}{l}
{x_{\text{b}}}\\
{y_{\text{b}}}\\
{z_{\text{b}}}
\end{array}\right]=\left[{\begin{array}{ccc}
	1 & 0 & 0\\
	0 & {\cos\phi} & {\sin\phi}\\
	0 & {-\sin\phi} & {\cos\phi}
	\end{array}}\right]\left[{\begin{array}{c}
	{n_{1}}\\
	{n_{2}}\\
	{n_{3}}
	\end{array}}\right]\label{eq:2.10}
\end{equation}
By letting
\begin{equation}
{{\mathbf{R}}_{x}}\left(\phi\right)\triangleq\left[\begin{array}{ccc}
1 & 0 & 0\\
0 & \cos\phi & \sin\phi\\
0 & -\sin\phi & \cos\phi
\end{array}\right]\label{eq:2.11}
\end{equation}
one has
\begin{equation}
\left[\begin{array}{l}
{x_{\text{b}}}\\
{y_{\text{b}}}\\
{z_{\text{b}}}
\end{array}\right]={\mathbf{R}_{x}}\left(\phi\right)\left[{\begin{array}{c}
	{n_{1}}\\
	{n_{2}}\\
	{n_{3}}
	\end{array}}\right]={\mathbf{R}_{x}}\left(\phi\right){\mathbf{R}_{y}}\left(\theta\right)\left[\begin{array}{l}
{k_{1}}\\
{k_{2}}\\
{k_{3}}
\end{array}\right]={\mathbf{R}_{x}}\left(\phi\right){\mathbf{R}_{y}}\left(\theta\right){\mathbf{R}_{z}}\left(\psi\right)\left[{\begin{array}{c}
	{x_{\text{g}}}\\
	{y_{\text{g}}}\\
	{z_{\text{g}}}
	\end{array}}\right].\label{eq:2.12}
\end{equation}
Therefore, the rotation matrix ${{\mathbf{R}}_{\text{{b}/{g}}}}$,
which represents the rotation from the ground frame to the body frame,
can be expressed as
\begin{equation}
%\begin{aligned}{l}
\begin{aligned}
{{\mathbf{R}}_{\text{{b}/{g}}}}&={{\mathbf{R}}_{x}}\left(\phi\right){{\mathbf{R}}_{y}}\left(\theta\right){{\mathbf{R}}_{z}}\left(\psi\right)\\
&=\left[\begin{array}{ccc}
1 & 0 & 0\\
0 & \cos\phi & \sin\phi\\
0 & -\sin\phi & \cos\phi
\end{array}\right]\left[\begin{array}{ccc}
\cos\theta & 0 & -\sin\theta\\
0 & 1 & 0\\
\sin\theta & 0 & \cos\theta
\end{array}\right]\left[\begin{array}{ccc}
\cos\psi & \sin\psi & 0\\
-\sin\psi & \cos\psi & 0\\
0 & 0 & 1
\end{array}\right]\\
&=\left[\begin{array}{ccc}
\cos\theta\cos\psi & \cos\theta\sin\psi & -\sin\theta\\
\sin\theta\sin\phi\cos\psi-\sin\psi\cos\phi & \sin\psi\sin\theta\sin\phi+\cos\psi\cos\phi & \sin\phi\cos\theta\\
\sin\theta\cos\phi\cos\psi+\sin\psi\sin\phi & \sin\psi\sin\theta\cos\phi-\cos\psi\sin\phi & \cos\phi\cos\theta
\end{array}\right]
\end{aligned}\label{2.13}
\end{equation}
The rotation from the ground frame to the body frame is composed of
three elemental steps, which can be represented as
\begin{equation}
\left[{\begin{array}{c}
	{x_{\text{g}}}\\
	{y_{\text{g}}}\\
	{z_{\text{g}}}
	\end{array}}\right]\begin{array}{c}
\\
\\
\underrightarrow{{\mathbf{R}_{z}}\left(\psi\right)}
\end{array}\left[\begin{array}{c}
{k_{1}}\\
{k_{2}}\\
{k_{3}}={\text{z}_{\text{g}}}
\end{array}\right]\begin{array}{c}
\\
\underrightarrow{{\mathbf{R}_{y}}\left(\theta\right)}\\
\\
\end{array}\left[{\begin{array}{c}
	{n_{1}}\\
	{{n_{2}}={k_{2}}}\\
	{n_{3}}
	\end{array}}\right]\begin{array}{c}
\underrightarrow{{\mathbf{R}_{x}}\left(\phi\right)}\\
\\
\\
\end{array}\left[\begin{array}{c}
{x_{\text{b}}}={n_{1}}\\
{y_{\text{b}}}\\
{z_{\text{b}}}
\end{array}\right].\label{eq:asd}
\end{equation}

In the docking process, the movement of the receiver needs to be converted
to the tanker frame for calculating, so it is necessary to know the
conversion relationship between the tanker frame and the receiver
frame. Since we have previously stated that the tanker frame is equivalent
to an inertial system whose direction is consistent with the ground
frame, we can directly describe the relationship between the two frames
by the attitude angle of the receiver. Here, the attitude of the receiver
is defined as ${{\mathbf{\Theta}}_{\text{r}}}={{\mathbf{\Theta}}_{\text{r/g}}}={{\mathbf{\Theta}}_{\text{r/t}}}=\left[\begin{array}{ccc}
{\theta}_{\text{r}} & {\phi}_{\text{r}} & {\psi}_{\text{r}}\end{array}\right]^{\text{T}}$, where ${{\theta}_{\text{r}}},{{\phi}_{\text{r}}},{{\psi}_{\text{r}}}$
represent the pitch angle, the roll angle and the yaw angle, respectively.
Then the rotation matrix can be expressed as
\begin{equation}
{{\mathbf{R}}_{\text{r/t}}}\left({{\mathbf{\Theta}}_{\text{r}}}\right)\text{=}\left[\begin{array}{ccc}
\cos{{\theta}_{\text{r}}}\cos{{\psi}_{\text{r}}} & \cos{{\theta}_{\text{r}}}\sin{{\psi}_{\text{r}}} & -\sin{{\theta}_{\text{r}}}\\
\sin{{\theta}_{\text{r}}}\cos{{\psi}_{\text{r}}}\sin{{\phi}_{\text{r}}}-\sin{{\psi}_{\text{r}}}cos{{\phi}_{\text{r}}} & \sin{{\theta}_{\text{r}}}\sin{{\psi}_{\text{r}}}\sin{{\phi}_{\text{r}}}+\cos{{\psi}_{\text{r}}}\cos{{\phi}_{\text{r}}} & \cos{{\theta}_{\text{r}}}\sin{{\phi}_{\text{r}}}\\
\sin{{\theta}_{\text{r}}}\cos{{\psi}_{\text{r}}}\cos{{\phi}_{\text{r}}}+\sin{{\psi}_{\text{r}}}\sin{{\phi}_{\text{r}}} & \sin{{\theta}_{\text{r}}}\cos{{\psi}_{\text{r}}}\cos{{\phi}_{\text{r}}}-\cos{{\psi}_{\text{r}}}\sin{{\phi}_{\text{r}}} & \cos{{\theta}_{\text{r}}}\cos{{\phi}_{\text{r}}}
\end{array}\right].\label{eq:2.14}
\end{equation}
The corresponding coordinates satisfy the following conversion relationship
\begin{equation}
{{\mathbf{p}}^{\text{r}}}={{\mathbf{R}}_{\text{r/t}}}\left({{\mathbf{\Theta}}_{\text{r}}}\right){{\mathbf{p}}^{\text{t}}}+\mathbf{p}_{\text{t/r}}^{\text{r}}\label{eq:2.15}
\end{equation}
or
\begin{equation}
{{\mathbf{p}}^{\text{t}}}=\mathbf{R}_{\text{r/t}}^{\text{T}}\left({{\mathbf{\Theta}}_{\text{r}}}\right){{\mathbf{p}}^{\text{r}}}+\mathbf{p}_{\text{r/t}}^{\text{t}}\label{eq:2.16}
\end{equation}
where $\mathbf{p}_{\text{t/r}}^{\text{r}}$ and $\mathbf{p}_{\text{r/t}}^{\text{t}}$
represent the relative position vector of the tanker and the receiver,
respectively.

The rotation from the wind frame to the body frame of an aircraft
is presented as follows. It takes two steps to transform vectors from
the wind frame ${{o}_{\text{w}}}-{{x}_{\text{w}}}{{y}_{\text{w}}}{{z}_{\text{w}}}$
to the body frame ${{o}_{\text{b}}}-{{x}_{\text{b}}}{{y}_{\text{b}}}{{z}_{\text{b}}}$. 

In the first step, we rotate the wind frame ${{o}_{\text{w}}}-{{x}_{\text{w}}}{{y}_{\text{w}}}{{z}_{\text{w}}}$
around the axis ${{o}_{\text{w}}}{{z}_{\text{w}}}$ by the side slip
angle $\beta$, which yields the stability frame ${{o}_{\text{s}}}-{{x}_{\text{s}}}{{y}_{\text{s}}}{{z}_{\text{s}}}$
as
\begin{equation}
\left[\begin{array}{l}
{x_{\text{s}}}\\
{y_{\text{s}}}\\
{z_{\text{s}}}
\end{array}\right]=\left[{\begin{array}{ccc}
	{\cos\beta} & {-\sin\beta} & 0\\
	{\sin\beta} & {\cos\beta} & 0\\
	0 & 0 & 1
	\end{array}}\right]\left[{\begin{array}{c}
	{x_{\text{w}}}\\
	{y_{\text{w}}}\\
	{z_{\text{w}}}
	\end{array}}\right].\label{eq:2.17}
\end{equation}
The rotation matrix from the wind frame to the stability frame is expressed as
\begin{equation}
{{\mathbf{R}}_{\text{{s}/{w}}}}\left(\beta\right)=\left[\begin{array}{ccc}
\cos\beta & -\sin\beta & 0\\
\sin\beta & \cos\beta & 0\\
0 & 0 & 1
\end{array}\right].\label{eq:2.18}
\end{equation}
The corresponding coordinates satisfy the following conversion relationship
\begin{equation}
{{\mathbf{p}}^{\text{s}}}={{\mathbf{R}}_{\text{s/w}}}\left(\beta\right){{\mathbf{p}}^{\text{w}}}\label{eq:2.19}
\end{equation}
or
\begin{equation}
{{\mathbf{p}}^{\text{w}}}=\mathbf{R}_{\text{s/w}}^{\text{T}}\left(\beta\right){{\mathbf{p}}^{\text{s}}}.\label{eq:2.20}
\end{equation}

In the second step, we rotate the obtained stability frame ${{o}_{\text{s}}}-{{x}_{\text{s}}}{{y}_{\text{s}}}{{z}_{\text{s}}}$
around the axis ${{o}_{\text{s}}}{{y}_{\text{s}}}$ by the angle of
attack $\alpha$, which yields the body frame ${{o}_{\text{b}}}-{{x}_{\text{b}}}{{y}_{\text{b}}}{{z}_{\text{b}}}$,
namely 
\begin{equation}
\left[\begin{array}{l}
{x_{\text{b}}}\\
{y_{\text{b}}}\\
{z_{\text{b}}}
\end{array}\right]=\left[{\begin{array}{ccc}
	{\cos\alpha} & 0 & {-\sin\alpha}\\
	0 & 1 & 0\\
	{\sin\alpha} & 0 & {\cos\alpha}
	\end{array}}\right]\left[\begin{array}{l}
{x_{\text{s}}}\\
{y_{\text{s}}}\\
{z_{\text{s}}}
\end{array}\right].\label{eq:2.21}
\end{equation}
The rotation matrix is expressed as
\begin{equation}
{{\mathbf{R}}_{\text{{b}/{s}}}}\left(\alpha\right)=\left[\begin{array}{ccc}
{\cos\alpha} & 0 & {-\sin\alpha}\\
0 & 1 & 0\\
{\sin\alpha} & 0 & {\cos\alpha}
\end{array}\right].\label{eq:2.22}
\end{equation}
The corresponding coordinates satisfy the following conversion relationship
\begin{equation}
{{\mathbf{p}}^{\text{b}}}={{\mathbf{R}}_{\text{b/s}}}\left(\alpha\right){{\mathbf{p}}^{\text{s}}}\label{eq:2.23}
\end{equation}
or
\begin{equation}
{{\mathbf{p}}^{\text{s}}}=\mathbf{R}_{\text{b/s}}^{\text{T}}\left(\alpha\right){{\mathbf{p}}^{\text{b}}}.\label{eq:2.24}
\end{equation}
According to Eq. (\ref{eq:2.17}) and Eq. (\ref{eq:2.21}), the following
equation can be obtained
\begin{equation}
\left[\begin{array}{l}
{x_{\text{b}}}\\
{y_{\text{b}}}\\
{z_{\text{b}}}
\end{array}\right]={{\bf {R}}_{{\rm {b/s}}}}\left(\alpha\right)\left[\begin{array}{l}
{x_{\text{s}}}\\
{y_{\text{s}}}\\
{z_{\text{s}}}
\end{array}\right]={{\bf {R}}_{{\rm {b/s}}}}\left(\alpha\right){{\bf {R}}_{{\rm {s/w}}}}\left(\beta\right)\left[{\begin{array}{c}
	{x_{\text{w}}}\\
	{y_{\text{w}}}\\
	{z_{\text{w}}}
	\end{array}}\right].\label{eq:2.25}
\end{equation}

In summary, the rotation matrix which represents the rotation from
the wind frame to the body frame, is expressed as
\begin{equation}
\begin{aligned}
%\begin{aligned}{lll}
{{\bf {R}}_{{\rm {b/w}}}}\left({\alpha,\beta}\right)&={{\bf {R}}_{{\rm {b/s}}}}\left(\alpha\right){{\bf {R}}_{{\rm {s/w}}}}\left(\beta\right)\\
&=\left[{\begin{array}{ccc}
	{\cos\alpha} & 0 & {-\sin\alpha}\\
	0 & 1 & 0\\
	{\sin\alpha} & 0 & {\cos\alpha}
	\end{array}}\right]\left[{\begin{array}{ccc}
	{\cos\beta} & {-\sin\beta} & 0\\
	{\sin\beta} & {\cos\beta} & 0\\
	0 & 0 & 1
	\end{array}}\right]\\
&=\left[{\begin{array}{ccc}
	{\cos\alpha\cos\beta} & {-\cos\alpha\sin\beta} & {-\sin\alpha}\\
	{\sin\beta} & {\cos\beta} & 0\\
	{\sin\alpha\cos\beta} & {-\sin\alpha\sin\beta} & {\cos\alpha}
	\end{array}}\right].
\end{aligned}\label{eq:2.26}
\end{equation}
The corresponding coordinates satisfy the following conversion relationship
\begin{equation}
{{\mathbf{p}}^{\text{b}}}={{\mathbf{R}}_{\text{b/w}}}\left(\alpha,\beta\right){{\mathbf{p}}^{\text{w}}}\label{eq:2.27}
\end{equation}
or
\begin{equation}
{{\mathbf{p}}^{\text{w}}}=\mathbf{R}_{\text{b/w}}^{\text{T}}\left(\alpha,\beta\right){{\mathbf{p}}^{\text{b}}}.\label{eq:2.28}
\end{equation}


\subsubsection{Air Velocity, Wind Velocity and Ground Velocity}

The aerodynamic forces and moments acting on an aircraft depend on
the airflow velocity (or simply \textquotedblleft air velocity\textquotedblright )
of the fuselage relative to the surrounding air (the wind velocity
should be considered) instead of the ground velocity. The ground velocity
and the air velocity are the same when there is no wind. However,
there are always many wind disturbances in the flight environment,
and we must carefully distinguish between air velocity $\mathbf{v}_{\text{{b}/{w}}}^{\text{g}}$
(the relative velocity with respect to the surrounding air) and the
ground velocity $\mathbf{v}_{\text{{b}/{g}}}^{\text{g}}$ (the relative
velocity with respect to the ground frame). Their relationship is
described as
\begin{equation}
\mathbf{v}_{\text{{b}/{w}}}^{\text{g}}=\mathbf{v}_{\text{{b}/{g}}}^{\text{g}}-\mathbf{v}_{\text{{w}/{g}}}^{\text{g}}\label{eq:2.29}
\end{equation}
where $\mathbf{v}_{\text{{w}/{g}}}^{\text{g}}$ represents the velocity
of the wind relative to the ground frame, namely the wind velocity.
Eq. (\ref{eq:2.29}) is described in the ground frame, which should
be converted into the body frame for convenience. First, by projecting
the ground velocity vector $\mathbf{v}_{\text{{b}/{g}}}^{\text{g}}$
into the body frame, we obtain vector $\mathbf{v}_{\text{{b}/{g}}}^{\text{b}}$
whose component form is expressed as
\begin{equation}
\mathbf{v}_{\text{{b}/{g}}}^{\text{b}}=\left[\begin{array}{c}
u\\
v\\
w
\end{array}\right].\label{eq:2.30}
\end{equation}
Secondly, by projecting the wind velocity vector $\mathbf{v}_{\text{{w}/{g}}}^{\text{g}}$
in Eq. (\ref{eq:2.29}) into the body frame, we obtain vector $\mathbf{v}_{\text{{w}/{g}}}^{\text{b}}$
whose component form is expressed as
\begin{equation}
\mathbf{v}_{\text{{w}/{g}}}^{\text{b}}=\left[\begin{array}{c}
{u}_{\text{w}}\\
{v}_{\text{w}}\\
{w}_{\text{w}}
\end{array}\right].\label{eq:3.31}
\end{equation}
Finally, projecting the air velocity vector $\mathbf{v}_{\text{{b}/{w}}}^{\text{g}}$
into the body frame yields $\mathbf{v}_{\text{{b}/{w}}}^{\text{b}}=\left[\begin{array}{ccc}
{u}_{\text{a}} & {v}_{\text{a}} & {w}_{\text{a}}\end{array}\right]^{\text{T}}$, whose relationship with $\mathbf{v}_{\text{{b}/{g}}}^{\text{b}}$
and $\mathbf{v}_{\text{{w}/{g}}}^{\text{b}}$ can be derived from
Eq. (\ref{eq:2.29}) as 
\begin{equation}
\mathbf{v}_{\text{{b}/{w}}}^{\text{b}}=\mathbf{v}_{\text{{b}/{g}}}^{\text{b}}-\mathbf{v}_{\text{{w}/{g}}}^{\text{b}}\label{eq:2.32}
\end{equation}
or written with the component form 
\begin{equation}
\left[{\begin{array}{c}
	{u_{\text{a}}}\\
	{v_{\text{a}}}\\
	{w_{\text{a}}}
	\end{array}}\right]=\left[\begin{array}{c}
u\\
v\\
w
\end{array}\right]-\left[{\begin{array}{c}
	{u_{\text{w}}}\\
	{v_{\text{w}}}\\
	{w_{\text{w}}}
	\end{array}}\right]=\left[{\begin{array}{c}
	{u-{u_{\text{w}}}}\\
	{v-{v_{\text{w}}}}\\
	{w-{w_{\text{w}}}}
	\end{array}}\right].\label{eq:2.33}
\end{equation}

Note that, the components ${{u}_{\text{a}}}$, ${{v}_{\text{a}}}$
and ${{w}_{\text{a}}}$ of $\mathbf{v}_{\text{{b}/{w}}}^{\text{b}}$
are often used to calculate the aerodynamic forces and moments when
building an aircraft dynamics model; The components $u$, $v$ and
$w$ of $\mathbf{v}_{\text{{b}/{g}}}^{\text{b}}$ are state variables
in the aircraft dynamics equations; the components ${{u}_{\text{w}}}$,
${{v}_{\text{w}}}$ and ${{w}_{\text{w}}}$ of $\mathbf{v}_{\text{{w}/{g}}}^{\text{b}}$
can be estimated from the flight data, which are usually generated
from the wind model in simulations. According to the conversion relationship
between the wind frame and the body frame in Eq. (\ref{eq:2.27}),
the following equation can be obtained 
\begin{equation}
\mathbf{v}_{\text{{b}/{w}}}^{\text{b}}={{\mathbf{R}}_{\text{b/w}}}\left(\alpha,\beta\right)\mathbf{v}_{\text{{b}/{w}}}^{\text{w}}\label{eq:2.34}
\end{equation}
where 
\[
\begin{array}{c}
\mathbf{v}_{\text{{b}/{w}}}^\text{w}=\left[\begin{array}{ccc}
{V}_{\text{a}} & 0 & 0\end{array}\right]^{\text{T}}\\
{{V}_{\text{a}}}=\left\Vert \mathbf{v}_{\text{{b}/{w}}}^{\text{b}}\right\Vert 
\end{array}
\]
in which ${{V}_{\text{a}}}$ is usually called the airspeed and can
be measured with an airspeed tube. Then Eq. (\ref{eq:2.34}) can be
written as
\begin{equation}
\left[{\begin{array}{c}
	{u_{\text{a}}}\\
	{v_{\text{a}}}\\
	{w_{\text{a}}}
	\end{array}}\right]=\left[{\begin{array}{ccc}
	{\cos\alpha\cos\beta} & {-\cos\alpha\sin\beta} & {-\sin\alpha}\\
	{\sin\beta} & {\cos\beta} & 0\\
	{\sin\alpha\cos\beta} & {-\sin\alpha\sin\beta} & {\cos\alpha}
	\end{array}}\right]\left[{\begin{array}{c}
	{V_{\text{a}}}\\
	0\\
	0
	\end{array}}\right]\label{eq:2.35}
\end{equation}
or 
\begin{equation}
\left[{\begin{array}{c}
	{u_{\text{a}}}\\
	{v_{\text{a}}}\\
	{w_{\text{a}}}
	\end{array}}\right]={V_{\text{a}}}\left[{\begin{array}{c}
	{\cos\alpha\cos\beta}\\
	{\sin\beta}\\
	{\sin\alpha\cos\beta}
	\end{array}}\right].\label{eq:2.36}
\end{equation}
Thus

\begin{equation}
{V_{\text{a}}}=\left\Vert {\mathbf{v}_{\text{b/w}}^{\text{b}}}\right\Vert =\sqrt{u_{\text{a}}^{2}+v_{\text{a}}^{2}+w_{\text{a}}^{2}}\label{eq:2.37}
\end{equation}

\begin{equation}
\alpha={\tan^{-1}}\left({\frac{{w_{\text{a}}}}{{u_{\text{a}}}}}\right)\label{eq:2.38}
\end{equation}

\begin{equation}
\beta={\sin^{-1}}\left({\frac{{v_{\text{a}}}}{{V_{\text{a}}}}}\right).\label{eq:2.39}
\end{equation}

Since the aerodynamic forces and moments of the aircraft are usually
determined by ${{V}_{\text{a}}}$, $\alpha$ and $\beta$, the above
expressions are essential for deriving the equations of motion of
the aircraft.

\subsubsection{Angular Velocity}

The rates of the change of roll, pitch and yaw angles with time are
represented by $\dot{\phi}$, $\dot{\theta}$ and $\dot{\psi}$ respectively,
which are critical for computing the Euler angles $\phi$, $\theta$,
$\psi$ in the flight dynamics simulations, but they are difficult
to measure directly in practice. The angle changing rate is usually
different from the angular velocity of an aircraft relative to the
ground frame, which can be measured by a rate gyro. 

As shown in Fig. \ref{Fig_2.10}, assuming that the angular velocity
of the aircraft is ${{\boldsymbol{\omega}}^{\text{b}}}={{\left[p\text{ }q\text{ }r\right]}^{\text{T}}}$,
then the relationship between angular velocity ${\boldsymbol{\omega}^{\text{b}}}$
and the changing rates $\dot{\phi}$, $\dot{\theta}$, $\dot{\psi}$
of Euler angles can be expressed as \cite{ducard2009fault}
\begin{equation}
{{\boldsymbol{\omega}}^{\text{b}}}=\dot{\psi}\cdot\mathbf{k}_{3}^{\text{b}}+\dot{\theta}\cdot\mathbf{n}_{2}^{\text{b}}+\dot{\phi}\cdot\mathbf{x}_{b}^{\text{b}}.\label{eq:2.40}
\end{equation}
Since 
\begin{equation}
\mathbf{n}_{2}^{\text{b}}={{\mathbf{R}}_{\text{{b}/{n}}}}\cdot\mathbf{n}_{2}^{\text{g}}={{\mathbf{R}}_{\text{{b}/{n}}}}\cdot{{\mathbf{y}}_{\text{g}}}\label{eq:2.41}
\end{equation}
\begin{equation}
\mathbf{k}_{3}^{\text{b}}={{\mathbf{R}}_{\text{{b}/{k}}}}\cdot\mathbf{k}_{3}^{\text{g}}={\mathbf{R}_{\text{{b}/{k}}}}\cdot{{\mathbf{z}}_{\text{g}}}\label{eq:2.42}
\end{equation}
\begin{equation}
\mathbf{x}_{\text{b}}^{\text{b}}=\left[\begin{array}{ccc}
1 & 0 & 0\end{array}\right]^{\text{T}}\label{eq:2.43}
\end{equation}
one has 
\begin{equation}
\mathbf{n}_{2}^{\text{b}}={{\mathbf{R}}_{\text{{b}/{n}}}}\cdot{{\mathbf{y}}_{\text{g}}}={{\left[\begin{array}{ccc}
		0 & \cos\phi & -\sin\phi\end{array}\right]}^{\text{T}}}\label{eq:2.44}
\end{equation}
\begin{equation}
\mathbf{k}_{3}^{\text{b}}={{\mathbf{R}}_{\text{{b}/{k}}}}\cdot{{\mathbf{z}}_{\text{g}}}={{\left[\begin{array}{ccc}
		-\sin\theta & \cos\theta\sin\phi & \cos\theta\cos\phi\end{array}\right]}^{\text{T}}}\label{2.45}
\end{equation}
where ${{\mathbf{R}}_{\text{{b}/{n}}}}={{\mathbf{R}}_{x}}\left(\phi\right)$,
${{\mathbf{R}}_{\text{{b}/{k}}}}={{\mathbf{R}}_{x}}\left(\phi\right){{\mathbf{R}}_{y}}\left(\theta\right)$.
The specific forms of ${{\mathbf{R}}_{y}}\left(\theta\right)$ and
${{\mathbf{R}}_{x}}\left(\phi\right)$ are given by (\ref{eq:2.7})
and (\ref{eq:2.11}). Combining (\ref{eq:2.40})-(\ref{2.45}) yields
\begin{equation}
\left[\begin{array}{c}
p\\
q\\
r
\end{array}\right]=\left[\begin{array}{ccc}
1 & 0 & -\sin\theta\\
0 & \cos\phi & \cos\theta\sin\phi\\
0 & -\sin\phi & \cos\theta\cos\phi
\end{array}\right]\left[\begin{array}{c}
\dot{\phi}\\
{\dot{\theta}}\\
{\dot{\psi}}
\end{array}\right]\label{eq:2.46}
\end{equation}
and then
\begin{equation}
\left[\begin{array}{c}
\dot{\phi}\\
{\dot{\theta}}\\
{\dot{\psi}}
\end{array}\right]=\left[\begin{array}{ccc}
1 & \tan\theta\sin\phi & \tan\theta\cos\phi\\
0 & \cos\phi & -\sin\phi\\
0 & {\sin\phi}/{\cos\theta} & {\cos\phi}/{\cos\theta}
\end{array}\right]\left[\begin{array}{c}
p\\
q\\
r
\end{array}\right].\label{eq:2.47}
\end{equation}
It can be observed from Eq. (\ref{eq:2.47}) that $\cos\theta$ appears
in the denominator, so the value of the fraction no longer has meaning
when $\cos\theta=0$ or $\theta=\pm\frac{\pi}{2}$, which is called
the singularity problem. Therefore, Eq. (\ref{eq:2.47}) can only
be used for aircraft modeling and control when the pitch angle $\theta$
is small.

\subsection{Direction Cosine Matrix method}

\subsubsection{Direction Cosine Matrix }

Direction cosine matrix, namely rotation matrix, is widely used in
frame transformation. According to the three-step rotation process
from the ground frame to the body frame presented in Eq. (\ref{eq:asd}),
the rotation matrix ${{\mathbf{R}}_{{\text{g}}/{\text{b}}}}$ can
be described by Euler angles $\phi$, $\theta$, $\psi$ as 
\begin{equation}
\begin{aligned}
%\begin{array}{l}
{{\mathbf{R}}_{{\text{{g}/{b}}}}}&=\mathbf{R}_{{\text{{g}/{b}}}}^{-1}\\
&=\mathbf{R}_{z}^{-1}\left(\psi\right)\mathbf{R}_{y}^{-1}\left(\theta\right)\mathbf{R}_{x}^{-1}\left(\phi\right)\\
&=\mathbf{R}_{z}^{\text{T}}\left(\psi\right)\mathbf{R}_{y}^{\text{T}}\left(\theta\right)\mathbf{R}_{x}^{\text{T}}\left(\phi\right)\\
&=\left[\begin{array}{ccc}
\cos\theta\cos\psi & \sin\theta\sin\phi\cos\psi-\sin\psi\cos\phi & \sin\theta\cos\phi\cos\psi+\sin\psi\sin\phi\\
\cos\theta\sin\psi & \sin\psi\sin\theta\sin\phi+\cos\psi\cos\phi & \sin\psi\sin\theta\cos\phi-\cos\psi\sin\phi\\
-\sin\theta & \sin\phi\cos\theta & \cos\phi\cos\theta
\end{array}\right].
\end{aligned}\label{eq:2.48}
\end{equation}
Conversely, Euler angles $\phi$, $\theta$, $\psi$ can be solved
with knowing the rotation matrix ${{\mathbf{R}}_{\text{{g}/{b}}}}$.
By letting the rotation matrix be
\begin{equation}
{{\mathbf{R}}_{\text{{g}/{b}}}}\triangleq\left[\begin{array}{ccc}
r_{11} & {{r}_{12}} & {{r}_{13}}\\
{{r}_{21}} & {{r}_{22}} & {{r}_{23}}\\
{{r}_{31}} & {{r}_{32}} & {{r}_{33}}
\end{array}\right]\label{eq:2.49}
\end{equation}
according to Eq.(\ref{eq:2.48}), one has
\begin{equation}
\left\{ \begin{array}{l}
\tan\psi=\frac{{{r}_{21}}}{{{r}_{11}}}\\
\sin\theta=-{{r}_{31}}\\
\tan\phi=\frac{{{r}_{32}}}{{{r}_{33}}}
\end{array}\right..\label{eq:2.50}
\end{equation}
Since the ranges of Euler angles are $\psi\in\left[-\pi,\pi\right]$,
$\theta\in\left[{-\pi}/{2},{\pi}/{2}\right],$ $\phi\in\left[{-\pi}/{2},{\pi}/{2}\right]$,
the solutions to Eq.(\ref{eq:2.50}) are
\begin{equation}
\left\{ \begin{array}{l}
\psi=\arctan\frac{{{r}_{21}}}{{{r}_{11}}}\\
\theta=\arcsin\left(-{{r}_{31}}\right)\\
\phi=\arctan\frac{{{r}_{32}}}{{{r}_{33}}}
\end{array}\right.\label{eq:2.51}
\end{equation}
where the value ranges of $\arctan\left(\cdot\right)$ and $\arcsin\left(\cdot\right)$
are $\left[{-\pi}/{2},{\pi}/{2}\right]$. 

\subsubsection{Relationship Between the Derivative of the Rotation Matrix and the
	Angular Velocity}

If only the rigid body\textquoteright s rotation (without translation)
is considered, then the derivative of a vector ${{\mathbf{r}}^{\text{g}}}\in{\mathbb{R}^{3}}$
with time satisfies 
\begin{equation}
\frac{\text{d}{{\mathbf{r}}^{\text{g}}}}{\text{d}t}={{\boldsymbol{\omega}}^{\text{g}}}\times{{\mathbf{r}}^{\text{g}}}\label{eq:2.52}
\end{equation}
where the symbol $\times$ represents the cross product of two vectors.
The physical significance of Eq. (\ref{eq:2.52}) can be vividly illustrated
by the circular motion in Fig. \ref{Fig_2.12}. 

The cross product of two vectors $\mathbf{a}\triangleq[\begin{array}{ccc}
a_{x} & {{a}_{y}} & {{a}_{z}}\end{array}]^{\text{T}}$ and $\mathbf{b}\triangleq[\begin{array}{ccc}
b_{x} & {{b}_{y}} & {{b}_{z}}\end{array}]^{\text{T}}$ is defined as \cite{murray2017mathematical}
\begin{equation}
\mathbf{a}\times\mathbf{b}\mathsf{=}{{\left[\mathbf{a}\right]}_{\times}}\mathbf{b}\label{eq:2.53}
\end{equation}
where 
\begin{equation}
{{\left[\mathbf{a}\right]}_{\times}}=\left[\begin{array}{ccc}
0 & -{{a}_{z}} & {{a}_{y}}\\
{{a}_{z}} & 0 & -{{a}_{x}}\\
-{{a}_{y}} & {{a}_{x}} & 0
\end{array}\right]\label{eq:2.54}
\end{equation}
is a skew-symmetric matrix. 

\begin{figure}
	\begin{centering}
		\includegraphics[width=0.4\textwidth]{Figures/Figs_Ch2/Fig_2\lyxdot 12}
		\par\end{centering}
	\caption{The derivative of a vector presented by a circular motion}
	
	\centering{}\label{Fig_2.12}
\end{figure}

Let $\mathbf{x}_{\text{b}}^{\text{b}},\mathbf{y}_{\text{b}}^{\text{b}},\mathbf{z}_{\text{b}}^{\text{b}}$
present the axial unit vectors of the body frame as
\begin{equation}
\begin{array}{c}
\mathbf{x}_{\text{b}}^{\text{b}}\triangleq\left[\begin{array}{ccc}
1 & 0 & 0\end{array}\right]^{\text{T}}\\
\mathbf{y}_{\text{b}}^{\text{b}}\triangleq\left[\begin{array}{ccc}
0 & 1 & 0\end{array}\right]^{\text{T}}\\
\mathbf{z}_{\text{b}}^{\text{b}}\triangleq\left[\begin{array}{ccc}
0 & 0 & 1\end{array}\right]^{\text{T}}
\end{array}
\end{equation}
then 
\[
\left[\begin{array}{ccc}
\mathbf{x}_{\text{b}}^{\text{b}} & \mathbf{y}_{\text{b}}^{\text{b}} & \mathbf{z}_{\text{b}}^{\text{b}}\end{array}\right]=\left[\begin{array}{ccc}
1 & 0 & 0\\
0 & 1 & 0\\
0 & 0 & 1
\end{array}\right]=\mathbf{I}_{3\times3}
\]
composes an identity matrix. Projecting vectors $\mathbf{x}_{\text{b}}^{\text{b}},\mathbf{y}_{\text{b}}^{\text{b}},\mathbf{z}_{\text{b}}^{\text{b}}$
from the body frame to the ground frame, one has
\begin{equation}
\begin{array}{c}
\mathbf{x}_{\text{b}}^{\text{g}}={\mathbf{R}}_{\text{{g}/{b}}}\cdot\mathbf{x}_{\text{b}}^{\text{b}}\\
\mathbf{y}_{\text{b}}^{\text{g}}={\mathbf{R}}_{\text{{g}/{b}}}\cdot\mathbf{y}_{\text{b}}^{\text{b}}\\
\mathbf{z}_{\text{b}}^{\text{g}}={\mathbf{R}}_{\text{{g}/{b}}}\cdot\mathbf{z}_{\text{b}}^{\text{b}}
\end{array}
\end{equation}
which yields
\begin{equation}
\begin{array}{ll}
\left[\begin{array}{ccc}
\mathbf{x}_{\text{b}}^{\text{g}} & \mathbf{y}_{\text{b}}^{\text{g}} & \mathbf{z}_{\text{b}}^{\text{g}}\end{array}\right] & ={\mathbf{R}}_{\text{{g}/{b}}}\cdot\left[\begin{array}{ccc}
\mathbf{x}_{\text{b}}^{\text{b}} & \mathbf{y}_{\text{b}}^{\text{b}} & \mathbf{z}_{\text{b}}^{\text{b}}\end{array}\right]\\
& ={\mathbf{R}}_{\text{{g}/{b}}}\cdot\mathbf{I}_{3\times3}\\
& ={\mathbf{R}}_{\text{{g}/{b}}}
\end{array}.
\end{equation}
According to Eq. (\ref{eq:2.52}), one has 
\begin{equation}
\begin{array}{ll}
\frac{\text{d}{{\mathbf{R}}_{\text{{g}/{b}}}}}{\text{d}t} & =\frac{\text{d}\left[\begin{array}{ccc}
	\mathbf{x}_{\text{b}}^{\text{g}} & \mathbf{y}_{\text{b}}^{\text{g}} & \mathbf{z}_{\text{b}}^{\text{g}}\end{array}\right]}{\text{d}t}=\left[\begin{array}{clc}
\frac{\text{d}\mathbf{x}_{\text{b}}^{\text{g}}}{\text{d}t} & \frac{\text{d}\mathbf{y}_{\text{b}}^{\text{g}}}{\text{d}t} & \frac{\text{d}\mathbf{z}_{\text{b}}^{\text{g}}}{\text{d}t}\end{array}\right]\\
& =\left[\begin{array}{ccc}
{\boldsymbol{\omega}}^{\text{g}}\times\mathbf{x}_{\text{b}}^{\text{g}} & {\boldsymbol{\omega}}^{\text{g}}\times\mathbf{y}_{\text{b}}^{\text{g}} & {\boldsymbol{\omega}}^{\text{g}}\times\mathbf{z}_{\text{b}}^{\text{g}}\end{array}\right]
\end{array}\label{eq:2.55}
\end{equation}
Since ${{\boldsymbol{\omega}}^{\text{g}}}={{\mathbf{R}}_{\text{{g}/{b}}}}\cdot{{\boldsymbol{\omega}}^{\text{b}}}$,
by using the properties of the cross product, one has
\begin{equation}
\begin{array}{ll}
{\boldsymbol{\omega}}^{\text{g}}\times\mathbf{x}_{\text{b}}^{\text{g}} & =\left({{\mathbf{R}}_{\text{{g}/{b}}}}\cdot{{\boldsymbol{\omega}}^{\text{b}}}\right)\times\left({{\mathbf{R}}_{\text{{g}/{b}}}}\cdot\mathbf{x}_{\text{b}}^{\text{b}}\right)\\
& ={{\mathbf{R}}_{\text{{g}/{b}}}}\cdot\left({\boldsymbol{\omega}}^{\text{b}}\times\mathbf{x}_{\text{b}}^{\text{b}}\right)\\
& ={{\mathbf{R}}_{\text{{g}/{b}}}}\cdot{\left[{{\boldsymbol{\omega}}^{\text{b}}}\right]}_{\times}\cdot\mathbf{x}_{\text{b}}^{\text{b}}
\end{array}.\label{eq:ss}
\end{equation}
where ${{\left[{{\boldsymbol{\omega}}^{\text{b}}}\right]}_{\times}}$
is the skew-symmetric form of ${{\boldsymbol{\omega}}^{\text{b}}}$.
Finally, by applying Eq. (\ref{eq:ss}) for ${\boldsymbol{\omega}}^{\text{g}}\times\mathbf{y}_{\text{b}}^{\text{g}}$
and $\mathbf{{\boldsymbol{\omega}}^{\text{g}}\times z}_{\text{b}}^{\text{g}},$
Eq. (\ref{eq:2.55}) is rewritten as 
\begin{equation}
\begin{array}{ll}
\frac{\text{d}{{\mathbf{R}}_{\text{{g}/{b}}}}}{\text{d}t} & =\left[\begin{array}{ccc}
{\boldsymbol{\omega}}^{\text{g}}\times\mathbf{x}_{\text{b}}^{\text{g}} & {\boldsymbol{\omega}}^{\text{g}}\times\mathbf{y}_{\text{b}}^{\text{g}} & {\boldsymbol{\omega}}^{\text{g}}\times\mathbf{z}_{\text{b}}^{\text{g}}\end{array}\right]\\
& ={{\mathbf{R}}_{\text{{g}/{b}}}}\cdot{\left[{{\boldsymbol{\omega}}^{\text{b}}}\right]}_{\times}\cdot\left[\begin{array}{ccc}
\mathbf{x}_{\text{b}}^{\text{b}} & \mathbf{y}_{\text{b}}^{\text{b}} & \mathbf{z}_{\text{b}}^{\text{b}}\end{array}\right]\\
& ={{\mathbf{R}}_{\text{{g}/{b}}}}\cdot{{\left[{{\boldsymbol{\omega}}^{\text{b}}}\right]}_{\times}}
\end{array}\label{2.56}
\end{equation}

The use of the rotation matrix can avoid the singularity problem.
However, since ${{\mathbf{R}}_{\text{{g}/{b}}}}$ has nine unknown
variables, the calculating quantity of solving Eq. (\ref{2.56}) is
large. 

\section{Chapter Summary}

This chapter mainly introduces several common frames in the aerial
refueling process, including the ground frame, the body frame, the
wind frame, the stability frame, the CFD system and the drogue equilibrium
position frame. Moreover, two methods for frame conversion are introduced,
that is, the Euler angle method and the direction cosine matrix method.
This chapter provides all the necessary knowledge for subsequent aerial
fueling modeling.
