
\chapter*{Preface}

Multicopter development began to flourish in the year 2013 because
of the high demand for aerial photography and the emergence of low-cost
multicopter products. New ideas, technologies, products, applications,
and investments associated with multicopters emerged one after another.
During the years 2016 and 2017, owing to an increase in market bubbles
and frequent incidents wherein civil aviation flights were threatened
by Unmanned Aerial Vehicles (UAVs), UAVs related industries suffered
a small decline. Meanwhile, many countries began to establish laws,
regulations, and standards for micro-small UAVs, open test bases,
organize the study of air traffic control systems for UAVs, and start
the processes of education and training. It can be said that the micro-small
UAV industries represented by multicopters have entered a new, orderly
phase with more opportunities for development. With the rise of 5G,
micro-small UAVs may become big data collection platforms at low altitudes,
which is expected to bring seven to ten times more business opportunities
for industry\footnote{Connected drones - a new perspective on the digital economy. Available
on https://cdn.microdrones.com/fileadmin/web/\_downloads/papers/Huawei\_whitepaper.pdf}.

The rapid development of multicopters is inseparable from the support
of open-source flight control systems. Whether based on open-source
flight control systems or fully independent development, developers
need to have a comprehensive grasp of multicopter design, modeling,
perception, control, and decision-making to produce multicopters that
deliver outstanding performance. However, much micro-small multicopter
research and development started with micro-small enterprises. Unlike
traditional aviation institutes, micro-small enterprises have disadvantages
such as fewer engineers, less experience, and fewer resources; thus,
chief and full-stack engineers are urgently needed. An important reason
for the shortage of such engineers is that textbooks and curriculum
have failed to keep up with the rapid development of multicopters.
Motivated by this, Beihang Reliable Flight Control Group (\url{http://rfly.buaa.edu.cn})
published a book in 2017 titled \textit{Introduction to Multicopter
Design and Control} in Springer Singapore.

Although we recognize that \textit{Introduction to Multicopter Design
and Control} can explain some theory problems for readers, the lack
of practical exercises means that it cannot help readers in deepening
their practical understanding. There is a certain gap that still exists
between the requirements of the industry and current engineering education,
which has motivated us to develop new tools and tutorials based on
multicopters for allowing the readers to apply the theory to practice.

Based on these considerations, the experiment of this book adopts
the following
\begin{enumerate}[(1) ]
\item the most widely used flight platform, multicopters, as a flight platform;
\item the most widely used open-source autopilot systems of aerial vehicles,
Pixhawk/PX4, as a control platform;
\item one of the most widely used programming languages in the field of
control engineering, MATLAB/Simulink, as a programming language.
\end{enumerate}
Based on the current advanced development of the \textit{Model-Based
Design} (MBD) process, the above three are closely linked. In addition
to the advancement of software, hardware, and development concepts,
the experiments consider coverage and differentiation. The book covers
eight tasks that include multicopter propulsion system design, dynamic
modeling, sensor calibration, state estimation and filter design,
attitude controller design, set-point controller design, semi-autonomous
control mode design, and failsafe logic design. These tasks cover
a comprehensive level of knowledge, and they can be completed by following
a progressive route. Each task consists of three step-by-step experiments
from shallow to deep, namely a basic experiment, an analysis experiment,
and a design experiment so that readers with different backgrounds
can benefit from them. Each experiment must be implemented in MATLAB/Simulink,
and simulation tests are carried out in the released software-in-the-loop
simulation platform. Furthermore, the readers can upload the control
algorithms to the Pixhawk autopilot through the automatic code generation
technology and form a closed-loop control system with a given real-time
simulator for Hardware-In-the-Loop (HIL) simulation tests. Throughout
the above process, readers can familiarize themselves with the basic
process of MBD, including the composition, mathematical model, and
control of a multicopter. Furthermore, readers can master a variety
of modern tools such as MATLAB/Simulink and FlightGear in the development
and computer simulation, a  HIL simulator, Pixhawk autopilot, and
a remote control transmitter in the HIL simulation test. The design
task also includes outdoor flight experiments to allow the readers
to experience the full development of a multicopter. Based on this,
the platform designed in this book won the gold award of National
Experimental Equipment Design Contest for College Automation Education
in China. Advanced development tools and processes allow micro-small
enterprises to realize new ideas rapidly, thereby overcoming the disadvantages
they face, which include fewer people, less experience, and fewer
resources. Furthermore, by lowering the learning threshold, more people
with other related backgrounds have opportunities to enter the aviation
field. 

This book can be considered a sister of \textit{Introduction to Multicopter
Design and Control}. To make this book self-contained, we have included
some theoretical parts of \textit{Introduction to Multicopter Design
and Control} that are required for the experiments in this book. The
process of preparing this book, which commenced in 2018, spanned nearly
two years. During this time, we first designed and implemented all
the experiments and then continually revised the book to present our
ideas as clearly as possible. Quan Quan designed the structure, experimental
process, and experiments of the entire book; Dai Xunhua designed the
experimental platform and development process; Wang Shuai conducted
all the experiments according to the experimental process. The successful
completion of this book is inseparable from the enthusiastic support
and help of the students of Beihang Reliable Flight Control Group.
First of all, we thank Yang Lanjiang, Liu Hao, and Ning Junqing for
participating in the experiments and assisting in the revision of
this book carefully and repeatedly. Secondly, we appreciate Liu Fanning
for the basic development of the experimental process and some experiments
that she conducted during her graduation project with our group. This
book was used for undergraduate students' curriculum design (B3I034140)
and graduate students' course (031574) partly, as well as in the courses
opened by a company, Tang Dynasty Robot (\url{https://www.chinarobot.com}).
We thank all the students who participated in the courses for their
helpful feedback.

The development of the experiments and tools in this book has been
expensive; therefore, we reserve the copyright for them. Experimental
codes and tools that we developed are available at \url{https://flyeval.com/course}
for personal use. No company or individual is allowed to sell the
codes and tools included in this book as educational products without
the authorization of Beihang Reliable Flight Control Group. Otherwise,
a legal response shall be pursued. For questions and licenses related
to this book, please contact us via \url{https://flyeval.com/course}.

\qquad{}\qquad{}\qquad{}\qquad{}\qquad{}\qquad{}\qquad{}\qquad{}\qquad{}\qquad{}\qquad{}\qquad{}\qquad{}\qquad{}Quan
Quan

\qquad{}\qquad{}\qquad{}\qquad{}\qquad{}\qquad{}\qquad{}\qquad{}\qquad{}\qquad{}\qquad{}\qquad{}\qquad{}\qquad{}Beihang
University  

\qquad{}\qquad{}\qquad{}\qquad{}\qquad{}\qquad{}\qquad{}\qquad{}\qquad{}\qquad{}\qquad{}\qquad{}\qquad{}\qquad{}Beijing,
China 

\qquad{}\qquad{}\qquad{}\qquad{}\qquad{}\qquad{}\qquad{}\qquad{}\qquad{}\qquad{}\qquad{}\qquad{}\qquad{}\qquad{}July
2019
