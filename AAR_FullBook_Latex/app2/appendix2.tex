
\chapter{How Teachers Use This Book}

This book provides eight experiments, each of which includes basic
experiments, analysis experiments, and design experiments. The code
for the basic and analysis experiments will be open on the designated
website while the code for design experiments will only be open to
the teachers who open the course. Students can first understand experimental
principles and basic code through basic experiments and analysis experiments,
based on which they can then carry out design experiments. To ensure
that different students have different design and experiment goals,
there are two solutions given as follows: modifying the goals in the
propulsion system design and modeling experiments for different students,
or opening new experiments. A detailed introduction is in the following.

\section{Modify the goals in the propulsion system design and modeling experiments
for different students}

To ensure that the design goals of different students are different,
a solution is given below.

\subsection{Modify the design experiment of the multicopter propulsion system}
\begin{enumerate}[(1) ]
\item Things to prepare\\
\url{https://flyeval.com/paper/}.
\item Objectives\\
1) Design a multicopter. The altitude is 0m, the local temperature
is 25~$^{\circ}$C, the weight of the airframe, autopilot, and accessories
is 1~kg totally, circumferential circle radius is smaller than 59.23\,in,
the total weight is lighter than (7+\textit{Y})~kg$\times$$\,$9.8$\,\textrm{m/s}^{2}$,
hover endurance is longer than 15~min, the hover throttle is less
than 65\% of the full-throttle. \\
2) List flight parameters and basic flight performance parameters
of the multicopters and compare the calculated results with that generated
by \url{https://flyeval.com/paper/}.
\end{enumerate}
Here, for students who select this course, we can modify the goals.
For example, dividing the last digit of a student ID by 4 obtains
the remainder \textit{X} = \{0: hexacopter, 1: coaxial hexacopter,
2: octocopter, 3: coaxial octocopter\} corresponding to different
types of multicopters. What is more, dividing the second-last digit
of the student ID by 4 obtains the remainder\textit{ Y}, which corresponds
to modifying the overall weight.

\subsection{Modify the design experiment for the multicopter modeling experiment}
\begin{enumerate}[(1) ]
\item Things to prepare\\
The well-designed multicopter configuration by ``Multicopter propulsion
system design experiment'' - ``Design Experiment'' and model parameters
provided by \url{https://flyeval.com/paper/}.
\item Objectives\\
Establish multicopter mathematical models, and build a complete multicopter
model on MATLAB/Simulink, and add a 3D multicopter model to FlightGear.
In terms of the attitude model, the quaternion model, the rotation
matrix model, or the Euler angle model can be used for students who
select this course. Dividing the third-last digit of their student
ID obtains the remainder \textit{Z }= \{0: quaternion model, 1: rotation
matrix model, 2: Euler angle model\}.
\end{enumerate}
With the two modified experiments above, it is difficult for two students
to have the same task. Based on the results from the two modified
experiments, the following multicopter attitude control design experiment
and multicopter position control design experiment are different correspondingly.
With the number of students increased, a new type of multicopters
can be added, such as a quadcopter, pentacopter, heptcopter, and so
on. If there are a large number of students, it is recommended that
they perform HIL simulation and indoor flight experiments, while outdoor
flight tests can be done only by teachers to show the process.

\section{Opening new experiments}

(1) Opening new experiments on filter design\textbf{ }

At present, this book only provides complementary filter and Kalman
filter experiments, and only for attitude estimation. As for the estimation
method, there are many new methods, such as unscented Kalman filter,
particle filter, and so on. In the aspect of the model and measurement
available, position estimation with GPS can be adopted as a new experiment,
for example. 

(2) Opening new experiments on multicopter attitude control design
and multicopter position control design

At present, this book only provides control experiments for general
readers with the ``principle of automatic control'' background.
For advanced and practical control methods, such as Active Disturbance
Rejection Control (ADRC) \cite{huang2014active} and model predictive
control \cite{grune2017nonlinear}, students can try in experiments,
especially in postgraduate related experimental courses.

(3) Other open experiments

In addition to the existing experiments, teachers can also open other
courses:
\begin{enumerate}[1) ]
\item Multicopter tracking controller design. The simplest task is to design
a given 4D trajectory (including time and position) tracking controller,
such as a circle (the head needs to point to the center of the circle)
or a number ``8''. 
\item Multicopter path following controller design. The simplest task is
to design path-following controllers for straight lines, circles,
and the number ``8''. Unlike tracking, the multicopter does not
need to fly to a given point on a path at a given time, as long as
its entire path follows the given path.
\item Multicopter obstacle avoidance controller design. Control a multicopter
to avoid a stationary spherical or stationary cylindrical obstacle,
or several obstacles. Besides, teachers can also design an experiment
which requires students to design a controller to avoid a geofence.
\item Multicopter area coverage decision-making design. Refer to the tasks
in Section\,13.1.12 of book\cite{quan2017introduction}. Consider
recharging, design an experiment which asks students for designing
a controller with its flight routes covering a whole rectangular farmland.
\end{enumerate}

